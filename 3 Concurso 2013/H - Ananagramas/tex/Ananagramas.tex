\documentclass[letter,10pt]{article}
\usepackage[utf8]{inputenc}
\usepackage[spanish, activeacute]{babel}
\usepackage{geometry}
\geometry{verbose,tmargin=0cm,bmargin=2cm,lmargin=2cm,rmargin=2cm,headheight=0cm,headsep=1cm,footskip=1cm}
\usepackage{graphicx}


%%%%%%%%%%%%%%%%%%%%%%%%%%%%%% Textclass specific LaTeX commands.
\newcommand{\lyxaddress}[1]{
\par {\raggedright #1
\vspace{1.4em}
\noindent\par}
}

%%%%%%%%%%%%%%%%%%%%%%%%%%%%%% User specified LaTeX commands.
\date{}

\begin{document}

\title{Problema H - Haciendo Ananagramas}

\includegraphics[scale=0.6]{logo} \hspace*{9.00cm}
\includegraphics[scale=0.5]{dsc} 
\bigskip
\begin{center}
	\Large Problema H - Haciendo Ananagramas
\end{center}

\begin{flushright}
Límite de tiempo: 3 segundos
\par\end{flushright}
\bigskip

\section*{Problema}

Muchos aficionados a los crucigramas suelen utilizar 
anagramas (grupos de palabras con las mismas letras en 
diferente orden) por ejemplo: AMOR, ROMA, OMAR, MORA,
RAMO, ARMO y MARO. Sin embargo, no todas las palabras cumplen
con este atributo, ya que no importa como trates de ordenar sus
letras no podrás formar otra palabra. Estas palabras se llaman
ananagramas, un ejemplo es SEXY.

Es obvio que la definición anterior depende del dominio con el
que nosotros estemos trabajando, tú podrías pensar que APARCAMIENTO
es un ananagrama, pero cualquier médico te podría desmentir 
fácilmente  ya que el te diría que existe METACARPIANO. Un posible
dominio podría ser todo el idioma español, pero esto podría acarrear 
algunos problemas. Nosotros podemos restringir el dominio, por ejemplo,
Geografía, en cuyo caso NEPAL se vuelve un anagrama relativo
(PANEL no esta en el mismo dominio), pero QUERETARO no lo es ya que
puede transformarse a TERRAQUEO.

Tú debes escribir un programa que lea un diccionario de un
dominio restringido de palabras y determinar todos los posibles
ananagramas relativos. Debes notar que las palabras de una sola
letra son, ipso facto, ananagramas relativos ya que no se pueden 
"reorganizar" de ningún otra forma. Nuestro diccionario no 
contendrá más de mil palabras. 

\subsection*{Entrada}

La entrada consiste en una serie de lineas. Ninguna línea
tendrá más de 80 caracteres de largo, pero puede contener
cualquier número de palabras. Las palabras consisten de 
hasta 20 caracteres (mayúsculas y minúsculas) letras y no se cortarán
entre lineas. Los espacios pueden aparecer libremente 
entre las palabras y por lo menos un espacio separa múltiples
palabras en la misma linea. Tenga en cuenta que las palabras
que contienen las mismas letras pero con diferente capitalización
se consideran anagramas entre si, por ejemplo CoNsErVaDorA y 
cOnvErsAdOra son anagramas. El archivo de entrada termina
con una linea consistente del símbolo \#.

\subsection*{Salida}

La salida consistirá de una serie de lineas. Cada línea consistirá
de una sola palabra que será un ananagrama relativo en el diccionario
de entrada. Las palabras deberán ser puestas en orden lexicográfico
(sensible a capitalización). Se asegura que siempre habrá por lo menos
unananagrama relativo.

\subsection*{Entrada ejemplo}
\noindent \texttt{ladder came tape soon leader acme RIDE lone Dreis peat}~\\
\texttt{ ScAlE orb  eye  Rides dealer  NotE derail LaCeS  drIed}~\\
\texttt{noel dire Disk mace Rob dries}~\\
\texttt{\#}~\\
\noindent 

$$$$
$$$$
$$$$
$$$$
$$$$

\subsection*{Salida Ejemplo}

\noindent \texttt{Disk}~\\
\texttt{NotE}~\\
\texttt{derail}~\\
\texttt{drIed}~\\
\texttt{eye}~\\
\texttt{ladder}~\\
\texttt{soon}~\\

\noindent \rule[0.5ex]{1\columnwidth}{1pt}


\lyxaddress{Waterloo Local Contest 2005 September 24 (Richard Krueger)}
\end{document}
