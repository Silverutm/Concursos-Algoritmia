\documentclass[letter,10pt]{article}
\usepackage{anysize}
\usepackage[utf8]{inputenc}
\usepackage[spanish, activeacute]{babel}
\usepackage{geometry}
\geometry{verbose,tmargin=0cm,bmargin=2cm,lmargin=2cm,rmargin=2cm,headheight=0cm,headsep=1cm,footskip=1cm}
\usepackage{graphicx}


%%%%%%%%%%%%%%%%%%%%%%%%%%%%%% Textclass specific LaTeX commands.
\newcommand{\lyxaddress}[1]{
\par {\raggedright #1
\vspace{1.4em}
\noindent\par}
}

%%%%%%%%%%%%%%%%%%%%%%%%%%%%%% User specified LaTeX commands.
\date{}

\begin{document}

\title{Problema B - Buscando Oro}

\includegraphics[scale=0.7]{logo} \hspace*{9.00cm}
\includegraphics[scale=0.5]{dsc} 
\bigskip
\begin{center}
	\Large Problema B - Buscando Oro
\end{center}

\begin{flushright}
Límite de tiempo: 3 segundos
\par\end{flushright}
\bigskip

\section*{Problema}

Has descubierto la Ciudad de Oro. Como te encanta el oro, has empezado a recolectarlo. Pero hay tanto oro que te estás cansando mientras lo recolectas, así que quieres saber cuál es el mínimo esfuerzo necesario para obtener todo el oro.

La ciudad está descrita con una cuadrícula $2D$, donde tu posición inicial está marcada con una '\textbf{x}'. Un espacio vacío se denotará con un '\textbf{.}'. Y los lugares que contienen oro se denotarán con una '\textbf{g}'. En cada movimiento puedes desplazarte a todos los $8$ lugares adyacentes dentro de la ciudad.

\subsection*{Entrada}

La primera línea será un entero $T$ ($1 \leq T \leq 100$), que denota el número de casos.
Cada caso empezará con dos enteros, $m$ y $n$ ($0 < m, n \leq 20$) denotando los renglones y columnas de la ciudad, respectivamente. Las siguientes $m$ líneas contienen $n$ caracteres que describen las ciudad. Habrá sólo una '$\textbf{x}$'  en la ciudad y a lo más 15 posiciones de oro.

\subsection*{Salida}

Para cada caso de prueba debes imprimir un entero indicando el mínimo número de pasos requeridos para recolectar todo el oro de la ciudad y regresar a la posición inicial.

\subsection*{Entrada ejemplo}

\noindent \texttt{2}~\\
\texttt{5 5}~\\
\texttt{x....}~\\
\texttt{g....}~\\
\texttt{g....}~\\
\texttt{.....}~\\
\texttt{g....}~\\
\texttt{5 5}~\\
\texttt{x....}~\\
\texttt{g....}~\\
\texttt{g....}~\\
\texttt{.....}~\\
\texttt{.....}~\\
\noindent 

\subsection*{Salida Ejemplo}

\noindent \texttt{Case 1: 8}~\\
\texttt{Case 2: 4}~\\

\noindent \rule[0.5ex]{1\columnwidth}{1pt}


\lyxaddress{LightOJ Online Judge}
\end{document}
