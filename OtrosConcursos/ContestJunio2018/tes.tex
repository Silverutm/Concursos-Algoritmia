\chapter*{Problema A - A ver}
Al profesor F acaba de descubrir una nueva serie, la cual es como sigue:

\[   
f(n) = 
     \begin{cases}
       n &\quad \text{si } 1 \leq n \\
       f(n - 1) + f(n-2) &\quad \text{si no}
     \end{cases}
\]

Justo cuando se disponia a darte el nombre de la serie fue interrumpido y le dijeron que la serie ya existe y que se llama de \textit{Fibonacci}, y como al parecer saben demasiado, les ha pedido resolver la siguiente suma:

$$ \sum_{i=0}^n f^2(i) $$

\subsection*{Entrada}
La primera línea de la entrada contendrá un número $T$, el número de casos. Luego vendrán $T$ líneas -una para
cada caso; cada una estará compuesta por un número $n$,
entero positivo.

\subsection*{Salida}
Para cada caso, imprime el valor de la suma, como la respuesta puede ser muy grande imprimela modulo $188888881$

\subsection*{Límites de los conjuntos de datos}

\begin{itemize}
    \item Pequeño: $ 1 \leq T \leq 10^3$, $ 1 \leq n \leq 10^2$   $\quad \;\;\;\;\;$ $20$ puntos.
    \item Mediano: $ 1 \leq T \leq 10^3$, $ 1 \leq n \leq 10^5$   $\quad \;\;\;\;\;$ $20$ puntos.
    \item Grande: $ 1 \leq T \leq 10^3$, $ 1 \leq n \leq 10^{18}$   $\quad \;\;\;\;\;$ $30$ puntos.
\end{itemize}

\begin{multicols}{2}
\subsection*{Entrada Ejemplo}

\begin{verbatim}
1
2
\end{verbatim}

\columnbreak

\subsection*{Salida Ejemplo}
\begin{verbatim}
2
\end{verbatim}
\end{multicols}


