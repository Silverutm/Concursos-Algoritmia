\chapter*{Problema B - Bases, Bases Everywhere}

¿No les pasa que cuando se transportan de un lugar a otro
buscan relaciones en todos los números que ven por ahí?
A mí sí, y les quiero compartir mi obsesión.

Les voy a explicar cuál es la dinámica de hoy: Tomo un
número y sumo sus dígitos, luego tomo el número resultante
y sumo sus dígitos. Repito este proceso hasta obtener un
sólo dígito. 

Cuando me di cuenta de que era muy fácil comencé a buscar 
cadenas de números con las letras, con el propósito de 
tener números en bases 11, 12, 13, 14, 15 y 16. A éstos
números les aplico el mismo proceso. Así, tú debes decirme
cuál es el caracter que me quedó al final.



\subsection*{Entrada}
La primera línea contendrá un número $T$, el número de 
casos de entrada. Posteriormente vendrán $T$ líneas, cada 
una con dos enteros positivos $n$ y $b$ separadas por un 
espacio. $n$ es el entero cuyos dígitos se suman y $b$ es
la base en que está.



\subsection*{Salida}
Para cada caso, imprime en una línea distinta y en base 
$b$, el caracter resultante de mi obsesivo proceso.



\subsection*{Límites de los conjuntos de datos}

\begin{itemize}
    \item Pequeño: $ 1 \leq T \leq 10^3 $, $ 1 \leq n
    \leq 10^8$   $\quad \;\;\;\;\;$ $80$ puntos.
    \item Grande: $ \;\, 1 \leq T \leq 10^{3}$, $ 1 
    \leq n \leq 10^{15} $ $\quad \quad$ $20$ puntos.
\end{itemize}



\begin{multicols}{2}

\subsection*{Entrada Ejemplo}

\begin{verbatim}
3
BC38A 14
395 11
A39B39 16
\end{verbatim}

\columnbreak

\subsection*{Salida Ejemplo}

\begin{verbatim}
5
7
F
\end{verbatim}

\end{multicols}