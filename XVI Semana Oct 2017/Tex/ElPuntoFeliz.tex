\chapter*{Problema E - El Punto Feliz}

En GUAPA queremos construir un patio para la diversión de 
Schnitzel, nuestra mascota. Planeamos dividir el espacio 
que tenemos de tal forma que podamos cubrir con pasto una
zona triangular, con el resto del área ya veremos luego 
qué hacemos.

Además, queremos construir un camino a partir de cada
vértice del triángulo de pasto hacia un punto en el patio
al que llamaremos ``El Punto Feliz".

Antes de empezar, debes saber que el patio de Schnitzel 
debe cumplir con las dos características más importantes
para ella:

\begin{enumerate}
    \item Ninguno de los lados del triángulo es paralelo
    al eje de las abscisas ni de las ordenadas.
    \item Los tres caminos entre los vértices del 
    triángulo y El Punto Feliz deben ser del mismo tamaño,
    sin importar que El Punto Feliz quede dentro o fuera
    del triángulo.
\end{enumerate}

Por último, debes saber que Schnitzel es un ser muy 
especial, por lo que sólo aceptará su nuevo patio si los
tres caminos de los que ya hablamos NO forman (entre dos
de ellos) un ángulo de 180°.



\subsection*{Entrada}

La primera línea de entrada es un número $T$ (el número de
casos de prueba). Siguen los $T$ casos: cada uno está
compuesto por tres pares ordenados $x,y$ de números reales 
positivos; cada par ordenado está en una línea distinta y 
es la coordenada de uno de los vértices del triángulo de 
pasto.



\subsection*{Salida}

Para cada caso de entrada deberás imprimir en una línea
las coordenadas del punto feliz, así como ``Aceptado"$\,$
en caso de que Schnitzel lo acepte, y ``No Aceptado"$\,$en 
caso contrario; ambos datos separados por un espacio.

Redondea tus respuestas a 4 decimales.



\subsection*{Límites de los conjuntos de datos}

\begin{itemize}
    \item Caso Único: $ 1 \leq T \leq 100 $, $ 1 \leq x, 
    y \leq 10^8 $   $\quad \;\;\;\;\;$ $100$ puntos.
\end{itemize}



\begin{multicols}{2}

\subsection*{Entrada Ejemplo}

\begin{verbatim}
2
1 1
5 3
3 5
0 0
6 6
8 4
\end{verbatim}

\columnbreak


\subsection*{Salida Ejemplo}

\begin{verbatim}
2.6667 2.6667 Aceptado
4.0000 2.0000 No Aceptado
\end{verbatim}

\end{multicols}