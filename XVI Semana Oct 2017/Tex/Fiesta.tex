\chapter*{Problema F - Fiesta de Ganadores}

Queremos hacer una fiesta para los ganadores del concurso
de programación, pero aquí entre nos, no tenemos mucho 
dinero, por lo que queremos rentar la menor cantidad de
sillas que sea posible. 

Cada persona invitada envía previamente el minuto en que
llega y el minuto en que se va (a partir del inicio de la 
fiesta, es decir, la fiesta inicia en el minuto 0), y lo 
cumple porque todos somos muy puntuales. Nosotros te 
daremos esos datos y tú determinarás cuál es la mayor 
cantidad de gente que puede haber en algún momento en la 
fiesta. Tu misión es decirnos dicha cantidad para que 
nosotros rentemos las sillas.



\subsection*{Entrada}
La primera línea contendrá un entero positivo $T$ (el 
número de casos de entrada). Cada caso iniciará con un
entero positivo $n$, que representa el número de personas 
invitadas a la fiesta. Luego vendrán $n$ líneas, cada una
compuesta por dos enteros $x, y$ tales que $x < y$,
separados por un espacio, $x$ para la hora de llegada y 
$y$ para la hora de partida de cada persona. Así, podemos
decir que cada persona permanece en la fiesta en el
intervalo $[x,y)$.



\subsection*{Salida}
Para cada caso, imprime en una línea distinta el número 
de sillas que debemos rentar.



\subsection*{Límites de los conjuntos de datos}

\begin{itemize}
    \item Pequeño: $ 1 \leq T \leq 10 $, $ 1 \leq n \leq
    100 $, $ 1 \leq x < y \leq 10^3$   $\quad \;\;\;\;\;$
    $50$ puntos.
    \item Mediano: $ 1 \leq T \leq 10 $, $ 1 \leq n \leq
    10^3 $, $ 1 \leq x < y \leq 10^{10}$ $\quad\;\;\;\;$ 
    $20$ puntos.
    \item Grande: $ \;\, 1 \leq T \leq 20$, $ 1 \leq n 
    \leq 10^4 $, $ 1 \leq x < y \leq 10^{12} $ $\quad 
    \quad \,$ $30$ puntos.
\end{itemize}



\begin{multicols}{2}

\subsection*{Entrada Ejemplo}

\begin{verbatim}
2
5
1 4
2 5
9 12
5 9
5 12
6
0 3
0 6
2 4
2 8
1 4
6 9
\end{verbatim}

\columnbreak

\subsection*{Salida Ejemplo}

\begin{verbatim}
2
5
\end{verbatim}

\end{multicols}