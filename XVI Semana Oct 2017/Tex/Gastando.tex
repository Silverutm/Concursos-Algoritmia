\chapter*{Problema G - Gastando en Regalos}

En GUAPA somos muy unidos, y para navidad queremos hacer
un intercambio de regalos. :D Como también somos muy 
previsores, hemos empezado a planear ya el intercambio.

A cada persona le asignamos un número e hicimos el 
acostumbrado sorteo con papelitos esperando que todos 
regalaran y recibieran algo. El problema es que cometimos 
un error al poner los papelitos en la caja, y es probable
que exista más de una persona asignada a regalarle algo a 
un humano, que exista un humano para el que nadie haya 
sido asignado o incluso que alguien tenga el papelito de su 
propio número.

Como el destinatario de nuestros regalos es, por el
momento, secreto, es tu deber determinar en qué casos
la asignación de las personas es correcto.



\subsection*{Entrada}
La primera línea estará compuesta únicamente por un
entero positivo $T$, que representa el número de casos que
se presentarán. Cada caso iniciará con una línea en la que 
se dará un número natural $n$ - el número de personas que
entrarán al intercambio. Finalmente, para cada caso se
darán $n$ líneas, cada una con dos números naturales $p$,
$q$, separados por un espacio, que representan, 
respectivamente, a la persona emisora y a la persona
receptora del regalo.



\subsection*{Salida}
Para cada caso, debes imprimir en una línea distinta 
``Intercambio existoso."$\,$si la asignación de las 
personas es correcta, o bien, ``Papelitos otra vez."$\,$si
no lo es.



\subsection*{Límites de los conjuntos de datos}

\begin{itemize}
    \item Caso Único: $ 1 \leq T \leq 100 $, $ 1 \leq n 
    \leq 10^3 $, $ 1 \leq p, q \leq n$   $\quad \;\;\;\;\;$ $100$ puntos.
\end{itemize}



\begin{multicols}{2}

\subsection*{Entrada Ejemplo}

\begin{verbatim}
4
3
1 1
2 3
3 2
4
1 2
2 1
3 4
4 3
4
1 2
2 3
3 2
4 1
3
1 2
2 3
3 1
\end{verbatim}

\columnbreak

\subsection*{Salida Ejemplo}

\begin{verbatim}
Papelitos otra vez.
Intercambio exitoso.
Papelitos otra vez.
Intercambio exitoso.
\end{verbatim}

\end{multicols}