\chapter*{Problema H - Hendrixland}

Mr. Malvavisco es un ser muy perezoso. Justo ahora está de
vacaciones en Hendrixland, y quiere recorrer todo lo que 
su pereza y su tiempo le permitan. Para eso te ha pedido
ayuda a ti. 

Mr. Malvavisco tiene en su poder una lista de los lugares
a los que le gustaría ir, representados, cada uno, con uno
de los primeros $V$ números naturales distinto, así como 
las conexiones inmediadas (las aristas) que existen entre 
ellos. Como ya te dije, Mr. Malvavisco es muy flojo, así 
que realmente no está dispuesto a perder su tiempo 
intentando llegar a un lugar que no es alcanzable desde 
alguno de los puntos de su lista, por lo que tú debes 
avisarle antes de que lo intente y se enoje con todos 
nosotros.



\subsection*{Entrada}
La primera línea, como siempre, será un entero $T$, el 
número de casos que recibirás. Luego, cada caso comenzará
con una línea compuesta por dos números naturales 
separados por un espacio $V$ y $A$, el número de lugares y
de conexiones inmediatas, respectivamente.
Posteriormente, para cada caso habrá $A$ líneas compuestas
de dos enteros $x, y$ tales que $1 \leq x, y \leq V$,
describiendo así que existe una arista que conecta a $x$
con $y$.



\subsection*{Salida}
Para cada caso, imprime en una línea distinta ``Go, Mr. 
Malva!"$\,$si es posible llegar de cada punto en la lista
de Mr. Malva a cualquier otro punto. En caso contrario,
imprime ``Go to sleep, Mr. Malva.".



\subsection*{Límites de los conjuntos de datos}

\begin{itemize}
    \item Pequeño: $ 1 \leq T \leq 10^2 $, $ 1 \leq V \leq
    10^2 $, $ 1 \leq A \leq 10^3 $    $\quad \;\;\;\;\;$ 
    $30$ puntos.
    \item Mediano: $ 1 \leq T \leq 10^2 $, $ 1 \leq V \leq
    10^3 $, $ 1 \leq A \leq 10^4 $    $\quad \;\;\;\;\;$ 
    $30$ puntos.
    \item Grande: $ \;\, 1 \leq T \leq 10 $, $ 1 \leq V \leq
    10^4 $, $ 1 \leq A \leq 10^5 $  $\quad \quad \quad$ 
    $40$ puntos.
\end{itemize}



\begin{multicols}{2}

\subsection*{Entrada Ejemplo}

\begin{verbatim}
2
3 2
1 2
2 3
4 2
1 2
3 4
\end{verbatim}

\columnbreak

\subsection*{Salida Ejemplo}

\begin{verbatim}
Go, Mr. Malva!
Go to sleep, Mr. Malva.
\end{verbatim}

\end{multicols}



\subsubsection*{Nota:}
Considera las conexiones como bidireccionales, es decir,
si se puede llegar del punto $x$ al punto $y$, entonces
también es posible llegar del punto $y$ al punto $x$.