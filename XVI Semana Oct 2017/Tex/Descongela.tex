\chapter*{Problema D - Descongela a Moro}

En el planeta Moroni el día de hoy pasó un cometa. Moro,
su gobernante, ama verlos pasar, por lo que tuvo un 
excelente día. 

Se sabe que dicho cometa pasa exactamente cada 165 años, 
pero como Moro está obsesionado con los cometas, quiere
congelarse y que sus súbditos lo descongelen para 
presenciar cada visita del mismo.

Te han designado como el matemático del planeta, por lo
que, considerando que actualmente es el año 0 en el 
planeta Moroni, tu misión es identificar cuáles son los
años en los que deberán descongelar a Moro.



\subsection*{Entrada}
La primera línea contendrá un entetro positivo $T$ (el 
número de casos a leer). Luego, vendrán $T$ líneas, cada
una con un entero positivo $n$, que representa el año para
el que debes predecir si el cometa pasará o no, es decir,
si descongelarán o no a Moro.



\subsection*{Salida}
Para cada caso, deberás imprimir en una línea distinta
``Descongelar a mi rey."$\,$si el cometa pasará en ese 
año, o bien, ``Dejar descansar a mi rey."$\,$en caso
contrario.



\subsection*{Límites de los conjuntos de datos}

\begin{itemize}
    \item Pequeño: $ 1 \leq T \leq 10^3 $, $ 1 \leq n
    \leq 10^5$   $\quad \;\;\;\;\;$ $20$ puntos.
    \item Mediano: $ 1 \leq T \leq 10^3 $, $ 1 \leq n
    \leq 10^{15}$   $\quad \;\;\;$ $20$ puntos.
    \item Grande: $ \;\, 1 \leq T \leq 10^{2}$, $ 1 
    \leq n \leq 10^{50} $ $\quad \;\;\;$ $60$ puntos.
\end{itemize}



\begin{multicols}{2}

\subsection*{Entrada Ejemplo}

\begin{verbatim}
4
165
168
1650
3328
\end{verbatim}

\columnbreak

\subsection*{Salida Ejemplo}

\begin{verbatim}
Descongelar a mi rey.
Dejar descansar a mi rey.
Descongelar a mi rey.
Dejar descansar a mi rey.
\end{verbatim}

\end{multicols}