\chapter*{Problema C - Coloreando la FES}

Alrededor de la FES vamos a colocar diversos puntos
que pintaremos de algún color. Nosotros tendremos $k$
colores para pintarlos de la forma en que se nos antoje.
El chiste aquí es que, sin que tú los veas, nos digas 
cuántos vértices tiene la figura geométrica más grande 
que puedes asegurar que existe con el mismo color de 
vértices. (En este caso, entendemos como la figura más 
grande, la de mayor número de vértices.



\subsubsection*{Nota:}
No existen tres vértices colineales.



\subsection*{Entrada}
La primera línea será un número natural $T$ (el número 
de casos que tendrás que leer), seguido de $T$ líneas
conformadas por dos enteros positivos separados por un
espacio $n$ y $k$; el primero representa el número de
puntos que se coloquen y el segundo el número de colores
que tendremos para pintarlos.



\subsection*{Salida}
Para cada caso, deberás imprimir en una línea distinta 
el número de vértices que tiene la figura geométrica más
grande que se puede formar con los vértices del mismo 
color.



\subsection*{Límites de los conjuntos de datos}

\begin{itemize}
    \item Pequeño: $ 1 \leq T \leq 10^3 $, $ 1 \leq n,
    k \leq 10^4$   $\quad \;\;\;\;\;$ $80$ puntos.
    \item Grande: $ \;\, 1 \leq T \leq 10^3 $, $ 1 
    \leq n, k \leq 10^{15} $ $\quad \quad$ $20$ puntos.
\end{itemize}



\begin{multicols}{2}

\subsection*{Entrada Ejemplo}

\begin{verbatim}
5
6 5
6 2
6 6
5 2
4 6
\end{verbatim}

\columnbreak

\subsection*{Salida Ejemplo}

\begin{verbatim}
2
3
1
3
1
\end{verbatim}

\end{multicols}