\chapter*{Problema A - Ahorrando la Fatiga}

Para este problema te ahorraré la fatiga de leer y te diré 
que simplemente tienes que jugar a algo muy sencillo:
contar.

El juego consiste en que digas cuántas palabras de $x_i$
letras se pueden formar con los caracteres $a$, $b$ y $c$ 
sin que haya dos $a$'s juntas.

Para este juego, entendemos como palabra una cadena de 
caracteres, sin importar que tenga sentido o no en alguno
de los 5 idiomas que sabes.



\subsection*{Entrada}
La primera línea de la entrada contendrá un número $T$, 
el número de casos. Luego vendrá una lista de enteros 
positivos $x_i$, $1 \leq i \leq T$ - uno por cada caso, 
que representa el tamaño de las palabras que debes contar.



\subsection*{Salida}
Para cada caso debes imprimir cuántas palabras se pueden
formar con $x_i$ caracteres de la forma que te dije arriba.

¡Ah, una cosa más! Debes imprimir la salida con módulo 
$10^9+7$.



\subsection*{Límites de los conjuntos de datos}

\begin{itemize}
    \item Pequeño: $ 1 \leq T \leq 10 $, $ 1 \leq x_i
    \leq 10$   $\quad \;\;\;\;\;$ $40$ puntos.
    \item Mediano: $ 1 \leq T \leq 10 $, $ 1 \leq x_i
    \leq 15$   $\quad \;\;\;\;\;$ $20$ puntos.
    \item Grande: $ \;\, 1 \leq T \leq 10^{3}$, $ 1 
    \leq x_i \leq 10^5 $ $\quad \;\,$ $40$ puntos.
\end{itemize}



\begin{multicols}{2}

\subsection*{Entrada Ejemplo}

\begin{verbatim}
3
1
2
5
\end{verbatim}

\columnbreak

\subsection*{Salida Ejemplo}

\begin{verbatim}
3
8
164
\end{verbatim}

\end{multicols}