\chapter*{Problema D - DO NOT talk about FIGHT CLUB}

%\begin{center}
 % \begin{tabular}{ | l | l | l | }
  %  \hline
   % Tiepo Límite: 2u & Memoria Límite: 512mb & Código Fuente: %\texttt{galaxiascelulares.\{java|cpp|c|py\}} \\
    %\hline
  %\end{tabular}
%\end{center}


Felicidades. Has sido aceptado en el proyecto Mayhem. Aquí no se te asignan tareas en específico,
sin embargo, para mantener el crecimiento de la organización y derribar a la civilización moderna lo más pronto posible, debes realizar el mayor número de tareas de una lista que te dan.

Dadas $N$ actividades, cada una con su hora de inicio y conclusión, imprime el máximo conjunto de actividades que puedes realizar. No está de más decirte que sólo puedes realizar una tarea a la vez.

Y recuerda: ¡Tyler Durden Lives!


\subsection*{Entrada}

La primera línea de entrada es un entero $T$ $(1 \leq T \leq 20)$, el número de casos de prueba.
Cada caso de prueba se compone de $N+1$ líneas: En la primera se dará un entero positivo $N$, y en las restantes un par de números enteros $s_i$ y $f_i$, separados por un espacio, donde $(0 \leq s_i < f_i)$ y $(s_i + 1 \leq f_i \leq 1000)$.

$s_i$ representa la hora de inicio y $f_i$ es la hora en que concluye la actividad, podemos decir que la i-ésima activadad se realiza en el intervalo $[s_i, f_i)$, donde $i=1,2,...,N$.


\subsection*{Salida}

Para cada caso de prueba, imprime el conjunto máximo de actividades que puedes realizar en el orden que se realizarían, separando cada actividad con un espacio.
Debes separar los casos de prueba con un salto de línea.


\subsection*{Límites de los conjuntos de datos}

\begin{itemize}
    \item Pequeño: $ 1 \leq N \leq 10 $ $\quad \quad \;\;$ $20$ puntos.
    \item Mediano: $ 1 \leq N \leq 10^2 $ $\quad \quad$ $25$ puntos.
    \item Grande:  $\;\; 1 \leq N \leq 10^4 $ $\quad \quad$ $55$ puntos.
\end{itemize}

\begin{multicols}{2}
\subsection*{Entrada Ejemplo}
\begin{verbatim}
1
6
7 9
1 2
0 6
5 9
3 4
5 7

\end{verbatim}
\columnbreak
\subsection*{Salida Ejemplo}
\begin{verbatim}
4
\end{verbatim}
\end{multicols}

\subsection*{Explicación}
La secuencia máxima de tareas que puedes realizar es la siguiente:
$[1,2)$, $[3,4)$, $[5,7)$, $[7,9)$. Por lo tanto, la respuesta es $4$. 

