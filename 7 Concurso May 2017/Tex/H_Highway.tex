\chapter*{Problema H - Highway Planning}


En algún lugar de un gran país, olvidaron construir carreteras que conectaran a las $N$ ciudades que hay en él. 
Por motivos de movilidad y economía, el jefe de construcción ha decido que no debe haber carreteras de una ciudad a ella misma, no puede haber mas de un camino entre un par de ciudades,  
además de que no pueden haber pistas que formen un ciclo de longitud $3$, esto significa que si hay tres ciudades conectadas, la última no puede tener una carretera directa a la primera. 

Al mencionar estas especificaciones, el jefe de construcción te ha pedido ayuda para calcular el máximo número de caminos que se pueden construir, y loa pares de ciudades donde deben de construirse las carreteras. 

Por cuestiones de practicidad las ciudades están identificadas por un entero $i$ $(1 \leq i \leq N)$.

\subsection*{Entrada}

La entrada consta de un entero $N$, el cual representa el total de ciudades en aquel país.

\subsection*{Salida}

Deberás imprimir el total de caminos $C$ que se pueden construir. Seguid de $C$ líneas con los pares de ciudades que se van a conectar, en orden ascendente.

\subsection*{Limites de los conjuntos de datos}
\begin{itemize}
    \item Pequeño: $ 1\leq N \leq 10 $ $\quad \quad $ $20$ puntos.
    \item Mediano: $ 1\leq N \leq 10^3 $ $\quad \quad$ $25$ puntos.
    \item Grande: $1 \leq N \leq 10^{4} $ $\quad \;$ $55$ puntos.
\end{itemize}

\begin{multicols}{2}
\subsection*{Entrada Ejemplo}
\begin{verbatim}
5
\end{verbatim}
\columnbreak
\subsection*{Salida Ejemplo}
\begin{verbatim}
6
1 2
1 4
2 3
2 5
3 4
4 5
\end{verbatim}
\end{multicols}

\textbf{Nota:}

Na na na na na na na na na na na na na
