\chapter*{Problema F - Fibonaccis y Malvaviscos}

%\begin{center}
%  \begin{tabular}{ | l | l | l | }
%    \hline
%   Tiepo Límite: 2u & Memoria Límite: 512mb & Código Fuente: %\texttt{findingekt.\{java|cpp|c|py\}} \\
%    \hline
%  \end{tabular}
%\end{center}

Malva es un malvavisco parlante al que le obsesionan los juegos con los números, y a quien le encanta ponerle retos a sus amiguitos. Cuando éstos logran resolver sus problemas, les regala un iPad, pero cuando no lo logran, se burla de ellos. 

Este reto consiste en lo siguiente:

Malva dice un número $n$. Entonces tú debes obtener el $n$-ésimo término
de la serie de Fibonacci. Pero esto no acaba aquí. Malva nunca lo hace fácil
(porque no cuenta con tantos recursos para comprar tantos iPad). Una vez 
obtenido el $n$-ésimo término, debes sumar los dígitos del mismo. Si el número 
resultante tiene más de un dígito, debes volver a sumar todos sus dígitos. 
Así, repites este proceso hasta que tu resultado sea únicamente de un dígito. 

A Malva le gusta que sigan sus instrucciones al pie de la letra, pero eso no es importante para mí, quien le dirá a Malva si lograste o no pasar su reto, así que si usas un procedimiento distinto, tranquilo, él no lo sabrá.Por último, considera $f_1=1$ y $f_2=1$.


\subsection*{Entrada}

La primera línea en la entrada será el número de casos, seguido de $T$ líneas, una para cada caso, únicamente compuestas por el número natural $n$.


\subsection*{Salida}

Para cada caso, imprime en una línea distinto el dígito que resulta en el reto de Malva para el $n$ dado.


\subsection*{Límites de los conjuntos de datos}

\begin{itemize}
    \item Pequeño: $ 1 \leq T \leq 30 $, $ 1 \leq n \leq 40 $ $\quad \quad \quad \;\,$ $30$ puntos.
    \item Mediano: $ 1\leq T \leq 60 $, $ 1 \leq n \leq 90 $ $\quad \quad \quad \;\,$ $25$ puntos.
    \item Grande: $ \,\; 1\leq T \leq 10^5 $, $ 1 \leq n \leq 10^6 $ $\quad \quad$ $45$ puntos.
\end{itemize}

\begin{multicols}{2}
\subsection*{Entrada Ejemplo}
\begin{verbatim}
4
6
11
12
18
\end{verbatim}
\columnbreak
\subsection*{Salida Ejemplo}
\begin{verbatim}
8
8
9
1
\end{verbatim}
\end{multicols}

\subsection*{Notas:}
En el segundo caso, el 11avo elemento de la secuencia de Fibonacci es 89. Como tiene más de un dígito, sumamos 8+9=17. Luego, 17 también tiene más de un dígito, así que repetimos el proceso: 1+7=8. La salida finalmente es 8.