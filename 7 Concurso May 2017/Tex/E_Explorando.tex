\chapter*{Problema E - Explorando la casa}

%\begin{center}
%  \begin{tabular}{ | l | l | l | }
%    \hline
%    Tiepo Límite: 2u & Memoria Límite: 512mb & %Código Fuente: %\texttt{eslomismo.\{java|cpp|c|py\}} \\
%    \hline
%  \end{tabular}
%\end{center}

Schnitzel sigue explorando su casa ya que dentro de ella existen varios puntos importantes señalados por $p_i$, $i=1,2,...,m-1,m$. 
Y ahora trata de llegar del  punto $p_1=1$ al punto $p_m=m$, sólo que ahora su casa se ve un tanto diferente: entre algunos pares de puntos existen túneles que los conectan y que se pueden recorrer únicamente en una dirección. 
A Schnitzel no le gusta perder su tiempo, así que antes de intentarlo, quiere saber si, utilizando esos túneles, es posible llegar del punto $1$ al punto $m$.

Ayúdala con eso.

\subsection*{Entrada}

En la primera línea de entrada, encontrarás un entero positivo $T$ ($1 \leq T \leq 20)$, que representa el número de casos a continuación. Cada caso estará separado del siguiente por una línea en blanco. En cada caso, la primera línea contiene a dos enteros positivos $m$ y $n$,  separados por un espacio; las siguientes $n$ líneas
estarán compuestas de pares de enteros positivos distintos $x_i$, $y_i$, separadas también por un espacio, que significa que existe un túnel que va del punto $x_i$ al punto $y_i$, y qué únicamente puede recorrerse en esa dirección.


\subsection*{Salida}

Para cada caso, imprime en una linea distinta ``Corre, Schnitzel.'' si existe un camino del punto $1$ al punto $m$; en caso contrario, imprime ``No te molestes.".


\subsection*{Límites de los conjuntos de datos}

\begin{itemize}
    \item Pequeño: $ 1 \leq x_i,y_i\leq m, 2\leq m \leq 100$, $2 \leq n \leq 500 $ $\quad \quad $ $30$ puntos.
    \item Mediano: $ 1 \leq x_i,y_i\leq m, 2\leq m \leq 500$, $2 \leq n \leq 10^3 $ $\quad \quad \quad \;\;$ $35$ puntos.
    \item Grande: $\,\;\, 1 \leq x_i,y_i\leq m, 2\leq m \leq 10^3$, $2 \leq n \leq 10^4$ $\quad \quad \quad \;\;$ $35$ puntos.
\end{itemize}

\begin{multicols}{2}
\subsection*{Entrada Ejemplo}
\begin{verbatim}
2
5 5
1 2
1 4
2 3
4 1
5 4

5 6
1 2
1 4
2 3
4 1
4 5
5 4
\end{verbatim}
\columnbreak


\subsection*{Salida Ejemplo}
\begin{verbatim}
No te molestes.
Corre, Schnitzel.
\end{verbatim}
\end{multicols}

