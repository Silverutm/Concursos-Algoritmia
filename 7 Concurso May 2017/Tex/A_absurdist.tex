\chapter*{Problema A - Absurdist Dystopian Black Comedy Film}

%\begin{center}
 % \begin{tabular}{ | l | l | l | }
  %  \hline
   % Tiepo Límite: 2u & Memoria Límite: 512mb & Código Fuente: %\texttt{galaxiascelulares.\{java|cpp|c|py\}} \\
    %\hline
  %\end{tabular}
%\end{center}

La reina V de Sandyland ha establecido una ley muy extraña: No deben existir personas solteras. 
Si una persona está sola debe encontrar pareja lo más pronto posible o es convertida en una langosta.
Lo peor es que para que dos habitantes puedan formar una pareja deben ser compatibles; que sean compatibles significa que deben tener a lo más $k$ características diferentes.
\\Las características de cada persona se representan en una tupla de 0's y 1's: 0 si no tienen cierta característica y 1 en caso contrario. 
Las tuplas de dos personas se comparan y se cuentan cuántos valores entre ambas son diferentes, si el número de valores diferentes es mayor a $k$, no son compatibles y serán convertidos en langostas.
A la reina V, se le ocurrió que no es necesario tener la tupla completa de $0$'s y los $1$'s, sino un número $n$ en base $10$ que represente esa tupla con elementos binarios.
Por ejemplo, si el número que representa tu tupla es $19$ y el de la reina V es $9$, sus tuplas se verían como sigue:

\begin{table}[!h]
\centering
\label{my-label}
\begin{tabular}{l|l|l|l|l|l|}
\cline{2-6} 
Tú      & \textbf{1} & \textbf{0} & 0 & \textbf{1} & 1 \\ \cline{2-6} 
Reina V & \textbf{0} & \textbf{1} & 0 & \textbf{0} & 1 \\ \cline{2-6} 
\end{tabular}
\end{table}

Si la reina V establece $k=4$ para la prueba contigo, serás afortunado, pues únicamente tienen tres características distintas, así que te salvarás de ser una langosta y serás compatible con la reina V.

Tu tarea es verificar si dos personas son compatibles dados dos números naturales $n$, $m$, que representan las tuplas de cada uno y $k$ que establece cuántas características diferentes a lo más pueden tener.
 

\subsection*{Entrada}

La primera línea de entrada es un número $T$ $(1 \leq T \leq 100)$, el número de casos de prueba. Siguen $T$ casos de prueba: cada uno contiene una sola línea con $n$, $m$ y $k$ separados por un espacio, para $1 \leq k \leq 60$. 


\subsection*{Salida}

Para cada caso de prueba imprime ``Son compatibles, sean felices por siempre :)", si el número de características diferentes es menor o igual a $k$. 
De lo contrario imprime ``Langostas por siempre :(".

\subsection*{Límites de los conjuntos de datos}
\begin{itemize}
    \item Pequeño: $ 1 \leq n, m \leq 10^2 , 1 \leq k \leq 4 $   $\quad \;\;\;\;\;$ $20$ puntos.
    \item Mediano: $ 1 \leq n, m \leq 10^6 , 1 \leq k \leq 8 $  $\quad \;\;\;\;\;$ $25$ puntos.
    \item Grande:  $ \;\, 1 \leq n, m \leq 10^{18}, 1 \leq k \leq 32 $ $\quad \quad$ $55$ puntos.
\end{itemize}

\begin{multicols}{2}
\subsection*{Entrada Ejemplo}
\begin{verbatim}
4
9 19 4
10 15 2
29 15 1
1 7 1   

\end{verbatim}
\columnbreak
\subsection*{Salida Ejemplo}
\begin{verbatim}
Son compatibles, sean felices por siempre :)
Son compatibles, sean felices por siempre :)
Langostas por siempre :(
Langostas por siempre :(
\end{verbatim}
\end{multicols}

\textbf{Nota:}

En la cultura popular se cree que las langostas son casi inmortales y cuando consiguen una pareja se quedan con ella por siempre.

