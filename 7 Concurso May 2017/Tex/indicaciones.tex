\chapter*{}
\vspace{-1cm}
\begin{center}
    \Huge \bf Concurso de la XV Semana de MAC
\end{center}

\bigskip

\subsection*{Indicaciones:}

\begin{itemize}

	\item La entrada del programa se debe leer de la entrada estándar \texttt{stdin}.
    
    \item La salida de programa se debe imprimir a la salida estándar \texttt{stdout}.
    
    \item Trata de usar métodos de lectura y escritura rápida para tus problemas, puesto que existen algunos problemas cuyas entradas pueden ser muy grandes.
    
    \item Una unidad de tiempo de tiempo para cada lenguaje son: \texttt{C} $\rightarrow$ 1 segundo, \texttt{C++} $\rightarrow$ 1 segundo, \texttt{C++14} $\rightarrow$ 1 segundo, \texttt{Java7} $\rightarrow$ 2 segundos, \texttt{Java8} $\rightarrow$ 2 segundos, \texttt{Python2} $\rightarrow$ 5 segundos y \texttt{Python3} $\rightarrow$ 5 segundos. De tal manera que si un problema dice 2u, en \texttt{Java7} tendrá un tiempo límite de 4 segundos.
   	
    \item No imprimir cadenas como: ``Dame la entrada'', ``La salida es:''
    \begin{minted}{cpp}
		printf("Dame la entrada:");/*Esto no se debe hacer*/
	\end{minted}
    
    \item No usar bibliotecas no estándar
  	
    Ejemplo:
    
    \begin{minted}{cpp}
#include <conio.h> /*Esta librería no es estándar*/

int main()
{
     system("pause");//No usar esto
     getch();//Ni esto
    
     return 0;
}
    \end{minted}
    
    \item Los problemas deberán llamarse como se indica en ``Código Fuente'', por ejemplo, para el primer problema si está hecho en Java será: \texttt{alreves.java}.
    
    \item Recuerda que está prohibido accesar a cualquier recurso de internet que no sean los señalados explícitamente el día del concurso.
    
    \item El usar cualquier dispositivo electrónico ajeno al concurso durante el transcurso del mismo es motivo de descalificación.
        
\end{itemize}

\subsection*{Mensaje al competidor}
Todos los problemas presentados aquí son originales e inéditos, elaborados por el Grupo de Algoritmia Avanzada y Programación Competitiva de la Facultad. Agradecemos a la Jefatura de MAC por el apoyo prestado para la elaboración de este concurso, así como, al Departamento de Servicios de Cómputo del CeDeTec por su constante presencia, apoyo, y patrocinio. 

Esperamos que se diviertan resolviendo estos problema.

Si les gustó el reto, no duden en preguntarnos cómo unirse al grupo y asistir a la sesión donde les enseñaremos a resolver todos los problemas del concurso.

\bigskip

\begin{center}
    \Large \it Happy Coding!
\end{center}


\thispagestyle{empty}