\documentclass[letter,10pt]{article}
\usepackage[utf8]{inputenc}
\usepackage[spanish, activeacute]{babel}
\usepackage{geometry}
\geometry{verbose,tmargin=0cm,bmargin=2cm,lmargin=2cm,rmargin=2cm,headheight=0cm,headsep=1cm,footskip=1cm}
\usepackage{graphicx}


%%%%%%%%%%%%%%%%%%%%%%%%%%%%%% Textclass specific LaTeX commands.
\newcommand{\lyxaddress}[1]{
\par {\raggedright #1
\vspace{1.4em}
\noindent\par}
}

%%%%%%%%%%%%%%%%%%%%%%%%%%%%%% User specified LaTeX commands.
\date{}

\begin{document}

\title{Problema E - Ésto es fácil}

\includegraphics[scale=0.6]{logo} \hspace*{9.00cm}
\includegraphics[scale=0.5]{dsc} 
\bigskip
\begin{center}
    \Large Problema E - Ésto es fácil
\end{center}

\begin{flushright}
Límite de tiempo: 0.5 segundos
\par\end{flushright}
\bigskip

\section*{Problema}

Éste problema es fácil, tienes que buscar 2 números en una lista, tal que sumados den otro número.

\subsection*{Entrada}

La primera línea tendrá un número T (0 $<$ T $<=$ 5000) el número de casos de prueba.\\[0.1cm]
Las siguientes 3*T líneas contendrán los casos de prueba, cada caso de prueba tendrá 3 líneas, la primera línea un número n (1 $<$ n $<=$ 401) la cantidad de números que tendrá el arreglo, la segunda línea tendrá n números $K_i$ (0 $<$ i $<=$ n) (0 $<$ $K_i$ $<=$ 10000) los números sobre los cuáles tienes que buscar la pareja de números, y la tercera línea tendrá un número M (0 $<$ M $<=$ 20000) el número a buscar.

\subsection*{Salida}

Se imprimirán T líneas, una por cada caso de prueba, con la palabra $"$SI$"$ si es que existen 2 números que sumados den M, y $"$NO$"$ en caso contrario.

\subsection*{Entrada Ejemplo}
\begin{verbatim}
2 
3
1 2 3
4
5
6 7 8 9 10
11
\end{verbatim}

\subsection*{Salida Ejemplo}

\begin{verbatim}
SI
NO

\end{verbatim}

\noindent \rule[0.5ex]{1\columnwidth}{1pt}


\lyxaddress{Sergio Adrián Lagunas Pinacho - UNAM FESA}
\end{document}
