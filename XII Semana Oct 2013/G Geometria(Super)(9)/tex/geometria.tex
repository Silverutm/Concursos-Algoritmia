\documentclass[letter,10pt]{article}
\usepackage[utf8]{inputenc}
\usepackage[spanish, activeacute]{babel}
\usepackage{geometry}
\geometry{verbose,tmargin=0cm,bmargin=2cm,lmargin=2cm,rmargin=2cm,headheight=0cm,headsep=1cm,footskip=1cm}
\usepackage{graphicx}


%%%%%%%%%%%%%%%%%%%%%%%%%%%%%% Textclass specific LaTeX commands.
\newcommand{\lyxaddress}[1]{
\par {\raggedright #1
\vspace{1.4em}
\noindent\par}
}

%%%%%%%%%%%%%%%%%%%%%%%%%%%%%% User specified LaTeX commands.
\date{}

\begin{document}

\title{Problema G - Geómetra Hermann}

\includegraphics[scale=0.6]{logo} \hspace*{9.00cm}
\includegraphics[scale=0.5]{dsc} 
\bigskip
\begin{center}
	\Large Problema G - Geómetra Hermann
\end{center}

\begin{flushright}
Límite de tiempo: 3 segundos
\par\end{flushright}
\bigskip

\section*{Problema}

Durante el siglo XIX el matemático Hermann Minkowski investigó acerca de
un tipo de geometría no euclidiana, llamada geometría de taxi. En la geometría
de taxi la distacia entre dos puntos $T_1(x_1,y_1)$ y $T_2(x_2, y_2)$ es definida
como $D(T_1, T_2) = |x_1-x_2|+|y_1-y_2|$. Cualquier otra definición se define
igual que en la geometría euclidiana, incluyendo la definición del círculo: Un
círculo es el conjunto de todos los puntos en un plano a una distancia 
dada(radio) de un punto dado(centro del círculo). Nosotros estamos interesados
en la diferencia de las areas de dos círculos con radio $R$, uno de los cuales 
está dado en un espacio normal(euclideano) y el otro en una geometría de taxi.
Dicha tarea te ha sido encomendada a ti.
 

\subsection*{Entrada}

Cada línea tendrá un solo número entero $1 \leq R \leq 10000$.
Deberaś leer hasta el final del archivo.

\subsection*{Salida}

Para cada caso deberás imprimir dos  líneas, la primera contendrá el area
del círculo con radio $R$ en una geometría euclidiana y la segunda línea será
el area del círculo con radio $R$ en una geometría de taxi.Tu salida deberá ser redondeada a 4 lugares decimales.
Además puedes asumir de manera segura de usando valores flotantes de precisión doble y $\pi$ igual 3.141592653589793 será suficiente.

\subsection*{Entrada Ejemplo}

\begin{verbatim}
21
9384
887
\end{verbatim}

\subsection*{Salida Ejemplo}

\begin{verbatim}
1385.4424
882.0000
276646940.0487
176118912.0000
2471707.7105
1573538.0000
\end{verbatim}

\noindent \rule[0.5ex]{1\columnwidth}{1pt}


\lyxaddress{Edgar García Rodríguez-Grupo de Algoritmia Avanzada y Programación Competitiva}
\end{document}
