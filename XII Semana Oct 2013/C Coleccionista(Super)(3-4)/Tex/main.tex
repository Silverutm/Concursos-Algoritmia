\documentclass[letter,10pt]{article}
\usepackage[utf8]{inputenc}
\usepackage[spanish, activeacute]{babel}
\usepackage{geometry}
\geometry{verbose,tmargin=0cm,bmargin=2cm,lmargin=2cm,rmargin=2cm,headheight=0cm,headsep=1cm,footskip=1cm}
\usepackage{graphicx}


%%%%%%%%%%%%%%%%%%%%%%%%%%%%%% Textclass specific LaTeX commands.
\newcommand{\lyxaddress}[1]{
\par {\raggedright #1
\vspace{1.4em}
\noindent\par}
}

%%%%%%%%%%%%%%%%%%%%%%%%%%%%%% User specified LaTeX commands.
\date{}

\begin{document}

\title{Problema C - Coleccionista}

\includegraphics[scale=0.6]{logo} \hspace*{9.00cm}
\includegraphics[scale=0.5]{dsc} 
\bigskip
\begin{center}
    \Large Problema C - Coleccionista
\end{center}

\begin{flushright}
Límite de tiempo: 1 segundo
\par\end{flushright}
\bigskip

\section*{Problema}

Hernán es un joven apasionado por los videojuegos, le gustan de todo género, desde shooters, plataformas, carreras y estrategia  hasta llegar a los de aventura. Uno sus juegos favoritos es el de Pokémon. Pokémon es un juego en el que tu objetivo es conseguir todos los monstruos disponibles(llamados pokémon) y las medallas de todos los gimnasios para así demostrar que has vencido a los mejores.\\

Aunque a Hernán le encantan todas las versiones de este juego, desde Rojo y Azul hasta X y Y, él no siempre quiere capturar a todos los pokémon disponibles en algunas versiones, ya sea porque considera que algunos son débiles o pequeños, e incluso feos.\\

Debido a esto(y a que no le gusta hacer algo tan fácil), te ha pedido que le ayudes con una sencilla tarea.\\

Dados el número total de pokémon disponibles en cada versión y dos listas de pokémon, la primera sera la lista de los que ya posee y la segunda la lista de los que no le interesan, determina las siguientes dos cosas:

\begin{itemize}
\item Cuantos y cuales pokémon  le faltan por capturar (solo de los que le interesan).
\item Cuantos y cuales pokémon que posee puede cambiar, esto es, los pokémon que ya tiene y no le interesan.
\end{itemize}

Los pokémon se identificaran por enteros para facilitar su procesamiento.\\

Es sabido que Hernán no posee mas de una especie del mismo pokémon por versión.\\ 

\subsection*{Entrada}

La entrada consta de varios casos de prueba.

Cada caso consta de 3 líneas, la primera línea un entero $N$, $10<=N<=1000$, que indica el total de pokémon en la versión del juego.

La segunda línea contiene un entero $A$,$0<=A<=N$, el número de pokémon que Hernán ha capturado en esa versión, seguido por una lista de $A$ enteros que representan a estos pokémon. Todos estos numeros estan separados por un espacio.

La tercera línea contiene un entero $X$,$0<=X<=N$, el total de pokemon que no le interesan a Hernan, seguido por $X$ enteros que representan a los mismos. Todos estos numeros estan separados por un espacio.

Para facilitar su identificación, los pokémon estan representados por numeros desde $1$ hasta $N$.
%$$$$
%$$$$
%$$$$
%$$$$
%$$$$
%$$$$

\subsection*{Salida}

Para cada caso imprime dos líneas:
\begin{itemize}
\item La primera constará de un entero $F$ que representará el número de pokémon que le interesa capturar a Hernan y aún no tiene, seguido por $F$ enteros que representan estos pokémon impresos en orden ascendente.
\item La segunda constará de un entero $C$ que representará el número de pokémon que Hernán puede cambiar, seguido por $C$ enteros que serán los pokémon cambiables impresos en orden ascendente.
\end{itemize}
$$$$
$$$$
$$$$
$$$$
\subsection*{Entrada Ejemplo}
\begin{verbatim}
11
5 3 5 9 10 11 
6 2 3 4 5 9 11 
\end{verbatim}

\subsection*{Salida Ejemplo}

\begin{verbatim}
4 1 6 7 8
4 3 5 9 11
\end{verbatim}

\noindent \rule[0.5ex]{1\columnwidth}{1pt}


\lyxaddress{Maximiliano Vera Luna - Grupo de Algoritmia Avanzada y Programación Competitiva}
\end{document}
