\documentclass[letter,10pt]{article}
\usepackage[utf8]{inputenc}
\usepackage[spanish, activeacute]{babel}
\usepackage{geometry}
\geometry{verbose,tmargin=0cm,bmargin=2cm,lmargin=2cm,rmargin=2cm,headheight=0cm,headsep=1cm,footskip=1cm}
\usepackage{graphicx}


%%%%%%%%%%%%%%%%%%%%%%%%%%%%%% Textclass specific LaTeX commands.
\newcommand{\lyxaddress}[1]{
\par {\raggedright #1
\vspace{1.4em}
\noindent\par}
}

%%%%%%%%%%%%%%%%%%%%%%%%%%%%%% User specified LaTeX commands.
\date{}

\begin{document}

\title{Problema D - Dominó}

\includegraphics[scale=0.6]{logo} \hspace*{9.00cm}
\includegraphics[scale=0.5]{dsc} 
\bigskip
\begin{center}
    \Large Problema D - Dominó
\end{center}

\begin{flushright}
Límite de tiempo: 1 segundo
\par\end{flushright}
\bigskip

\section*{Problema}

Mauricio tiene $N$ piezas de domino en fila. Cada pieza se divide en dos partes iguales en tamaño, - la superior y la inferior -. Cada una de las partes contiene un número del 1 al 6.
A Mauricio le encantan los números pares, así que quiere que la suma de los números de las mitades superiores y la suma de los números de las mitades inferiores sea par.

Para lograr eso, Mauricio puede rotar las piezas de dominó 180 grados. Despues de una rotación las mitades cambian lugares. Esta acción toma 1 segundo. Ayuda a Mauricio a averiguar el tiempo mínimo que debe gastar girando piezas de dominó para que se cumpla lo que él quiere.


\subsection*{Entrada}

La primera línea contiene un entero $T(1<T<50)$, el número de casos de prueba.
Para cada uno de los siguientes $T$ casos, hay una línea que contiene un entero $N(1<=N<=100)$, el número de piezas de dominó que tiene Mau. Las siguientes $N$ líneas contienen dos enteros $X_{i},Y_{i},(1<=X_{i},Y_{i}<=6)$ cada una, separados por un espacio. El número $X_{i}$ es el que esta escrito inicialmente en la mitad superior del i-ésimo dominó, mientras que $Y_{i}$ es el que está escrito en la mitad inferior. 

%$$$$
%$$$$
%$$$$
%$$$$
%$$$$
%$$$$

\subsection*{Salida}

Para cada caso imprime solo una línea, esta deberá decir Caso T: seguido por el mínimo de segundos requeridos para realizar la tarea. T es el caso actual empezando a numerarlos desde 1. Si resulta imposible lograr tal tarea, imprime un '-1'(sin comilla).

\subsection*{Entrada Ejemplo}
\begin{verbatim}
2
2
4 2
6 4
1
2 3
\end{verbatim}

\subsection*{Salida Ejemplo}

\begin{verbatim}
Caso 1 :0
Caso 2 :-1
\end{verbatim}

\noindent \rule[0.5ex]{1\columnwidth}{1pt}


\lyxaddress{Caribbean Online Judge}
\end{document}
