\documentclass[letter,10pt]{article}
\usepackage[utf8]{inputenc}
\usepackage[spanish, activeacute]{babel}
\usepackage{geometry}
\geometry{verbose,tmargin=0cm,bmargin=2cm,lmargin=2cm,rmargin=2cm,headheight=0cm,headsep=1cm,footskip=1cm}
\usepackage{graphicx}


%%%%%%%%%%%%%%%%%%%%%%%%%%%%%% Textclass specific LaTeX commands.
\newcommand{\lyxaddress}[1]{
\par {\raggedright #1
\vspace{1.4em}
\noindent\par}
}

%%%%%%%%%%%%%%%%%%%%%%%%%%%%%% User specified LaTeX commands.
\date{}

\begin{document}

\title{Problema F -  Falsa Simulación}

\includegraphics[scale=0.6]{logo} \hspace*{9.00cm}
\includegraphics[scale=0.5]{dsc} 
\bigskip
\begin{center}
	\Large Problema F -  Falsa Simulación
\end{center}

\begin{flushright}
Límite de tiempo: 1 segundos
\par\end{flushright}
\bigskip

\section*{Problema}

El siguiente problema consta de una simulación muy concreta... ¿o no?.
Tendrán una matriz $M$ de tipo $1234$(filas) $\times 5678$(columnas). Está inicialmente llena con enteros
desde el $1, 2, \dots, 1234 \times 5678$ de menor a mayor por filas.
Existen 4 tipos de comandos:
\begin{itemize}
    \item ``R x y'' intercambia la $x$-ésima y $y$-ésima fila de $M$.
    \item ``C x y'' intercambia la $x$-ésima y $y$-ésima columna de $M$.
    \item ``Q x y'' imprime  $M(x,y)$.
    \item ``W z'' imprime $x$ e $y$ donde $z=M(x,y)$.
\end{itemize}

\subsection*{Entrada}

Te serán dadas una lista de comandos válidos. La entrada de termina con el fin de archivo.

\subsection*{Salida}

Para cada `W z'' imprime una línea con $x$ e $y$ separados por un espacio donde $z=M(x,y)$ y para cada ``Q x y'' imprime una línea con $M(x,y)$.

\subsection*{Entrada Ejemplo}

\begin{verbatim}
R 1 2
Q 1 1
Q 2 1
W 1
W 5679
C 1 2
Q 1 1
Q 2 1
W 1
W 5679
\end{verbatim}

$$$$
$$$$
$$$$
$$$$
$$$$
$$$$
$$$$
$$$$
$$$$
$$$$
$$$$

\subsection*{Salida Ejemplo}

\begin{verbatim}
5679
1
2 1
1 1
5680
2
2 2
1 2
\end{verbatim}

\noindent \rule[0.5ex]{1\columnwidth}{1pt}


\lyxaddress{Sphere Online Judge}
\end{document}
