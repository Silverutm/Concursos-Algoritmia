\documentclass[letter,10pt]{article}
\usepackage[utf8]{inputenc}
\usepackage[spanish, activeacute]{babel}
\usepackage{geometry}
\geometry{verbose,tmargin=0cm,bmargin=2cm,lmargin=2cm,rmargin=2cm,headheight=0cm,headsep=1cm,footskip=1cm}
\usepackage{graphicx}


%%%%%%%%%%%%%%%%%%%%%%%%%%%%%% Textclass specific LaTeX commands.
\newcommand{\lyxaddress}[1]{
\par {\raggedright #1
\vspace{1.4em}
\noindent\par}
}

%%%%%%%%%%%%%%%%%%%%%%%%%%%%%% User specified LaTeX commands.
\date{}

\begin{document}

\title{Problema C - Modulo}

\includegraphics[scale=0.6]{logo} \hspace*{9.00cm}
\includegraphics[scale=0.5]{dsc} 
\bigskip
\begin{center}
    \Large Problema C - Modulo
\end{center}

\begin{flushright}
Límite de tiempo: 1 segundo
\par\end{flushright}
\bigskip

\section*{Problema}

Dados dos enteros A y B, A modulo B es el residuo de dividir A entre B. Por ejemplo, los numeros 7, 14, 27 y 38 se convierten en 1, 2, 0 y 2, si les aplicamos modulo 3. Desarrolla un programa que acepte \textbf{N} números como entrada e imprima el total de números distintos de la entrada, si los numeros son considerados modulo 42.

\subsection*{Entrada}

La primera línea de entrada contiene un entero \textbf{N}, $10<=N<=500$. Las siguientes N lineas contienen un entero X,$0<=X<=1000$.
%$$$$
%$$$$
%$$$$
%$$$$
%$$$$
%$$$$

\subsection*{Salida}

Imprime en una sola línea el número de valores distintos cuando son considerados modulo 42.

\subsection*{Entrada Ejemplo}
\begin{verbatim}
10
39
40
41
42
43
44
82
83
84
85
\end{verbatim}

\subsection*{Salida Ejemplo}

\begin{verbatim}
6
\end{verbatim}

\noindent \rule[0.5ex]{1\columnwidth}{1pt}


\lyxaddress{Caribbean Online Judge}
\end{document}
