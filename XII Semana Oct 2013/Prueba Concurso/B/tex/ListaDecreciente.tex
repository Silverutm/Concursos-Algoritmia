\documentclass[letter,10pt]{article}
\usepackage[utf8]{inputenc}
\usepackage[spanish, activeacute]{babel}
\usepackage{geometry}
\geometry{verbose,tmargin=0cm,bmargin=2cm,lmargin=2cm,rmargin=2cm,headheight=0cm,headsep=1cm,footskip=1cm}
\usepackage{graphicx}


%%%%%%%%%%%%%%%%%%%%%%%%%%%%%% Textclass specific LaTeX commands.
\newcommand{\lyxaddress}[1]{
\par {\raggedright #1
\vspace{1.4em}
\noindent\par}
}

%%%%%%%%%%%%%%%%%%%%%%%%%%%%%% User specified LaTeX commands.
\date{}

\begin{document}

\title{Problema B - Lista en Orden Decreciente}

\includegraphics[scale=0.6]{logo} \hspace*{9.00cm}
\includegraphics[scale=0.5]{dsc} 
\bigskip
\begin{center}
	\Large Problema B - Lista en Orden Decreciente
\end{center}

\begin{flushright}
Límite de tiempo: 3 segundos
\par\end{flushright}
\bigskip

\section*{Problema}

Se te dará una lista de a lo más 1000 enteros y debes ordenarlos 
decrecientemente.

\subsection*{Entrada}

La primera de línea de entrada es N, el número de elementos en la lista.
En cada una de las siguientes $N$ lineas, hay un entero $N_i$, tal que 
$1 \leq N_i \leq 1000$.

\subsection*{Salida}

Debes imprimir los mismos $N$ números, uno por línea, pero ahora en 
orden decreciente.

\subsection*{Entrada Ejemplo}

\begin{verbatim}
3
15
35
1
\end{verbatim}

\subsection*{Salida Ejemplo}

\begin{verbatim}
35
15
1
\end{verbatim}

\noindent \rule[0.5ex]{1\columnwidth}{1pt}


\lyxaddress{Edgar García Rodríguez UNAM-FESA}
\end{document}
