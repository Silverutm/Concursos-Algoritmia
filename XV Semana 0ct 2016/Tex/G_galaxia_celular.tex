\chapter*{Problema G - Galaxias Celulares}

\begin{center}
  \begin{tabular}{ | l | l | l | }
    \hline
    Tiepo Límite: 2u & Memoria Límite: 512mb & Código Fuente: \texttt{galaxiascelulares.\{java|cpp|c|py\}} \\
    \hline
  \end{tabular}
\end{center}

Paquita Cabeza ha llegado a un nivel de conocimiento tal, que ha decidido observar la evolución de una población sin intervenir en su vida. Estos seres viven en un mundo cuadrado, semejante a una matriz, y viven, mueren y se reproducen según ciertas reglas que Paquita alcanzó a comprender. Paquita observa la población en el tiempo $T=0$, y se pregunta cómo quedará la población en el tiempo $T=t, t\geq 0$. Para pasar de un tiempo $T=t$ a $T= t+1$ se aplica la siguiente regla para todas las celdas dónde viven los seres.
\begin{itemize}
    \item Si un ser vivo tiene menos de dos vecinos vivos muere por soledad
    \item Si un ser esta vivo, continua con vida si tiene 2 o 3 vecinos vivos
    \item Si un ser esta vivo, muere si tiene más de 3 vecinos vivos.
    \item Si un ser esta muerto, revive si tiene exactamente 3 vecinos vivos
\end{itemize}

\subsection*{Entrada}

La primera línea de entrada será un número $C$ $(1 \leq  C\leq 10)$, que es la cantidad de casos de prueba. Cada caso de prueba empieza con una linea con dos numeros $f$ y $c$, el tamaño de la matriz donde viven los seres, después vendran f filas con c caracteres cada una, representando dicha matriz, cada uno de estos elementos es o ``.'' o ``*'', representando a un ser muerto o a un ser vivo, respectivamente. Los vecinos de una celda son las celdas que la rodean a los lados y en diagonal. \\
Después vendrá un número t, indicando el tiempo T al cual Paquita desea saber como quedará la población. \\
Te recomendamos copiar la entrada del archivo pdf.
\subsection*{Salida}
Para cada caso imprime una matriz de $f*c$,que muestre como quedo la población en el tiempo t.

\subsection*{Limites de los conjuntos de datos}
\begin{itemize}
    \item Único Caso: $ 1\leq f, c, t \leq 100 $ $\quad \quad$ $100$ puntos.
\end{itemize}

\newpage
\subsection*{Entrada Ejemplo}
\begin{verbatim}
1
22 88
........................................................................................
........................................................................................
........................................................................................
........................................................................................
........................................................................................
........................................................................*...............
.......................................................................*.......*.*......
......................................................................**..*.*..*........
..........................................................................**.**.........
......................................................................**.*..***.........
......................................................................****...**...*.....
......................................................................****...**...*.....
......................................................................**.*..***.........
..........................................................................**.**.........
......................................................................**..*.*..*........
.......................................................................*.......*.*......
........................................................................*...............
........................................................................................
........................................................................................
........................................................................................
........................................................................................
........................................................................................
100
\end{verbatim}

\subsection*{Salida Ejemplo}
\begin{verbatim}
........................................................................................
........................................................................................
........................................................................................
........................................................................................
........................................................................................
........................................................................................
.....................................***.......*........................................
.....................................**..*.**.**........................................
.......................................***....*.........................................
.....................................*..*.*...*.........................................
....................................*....*....*.........................................
....................................*....*....*.........................................
.....................................*..*.*...*.........................................
.......................................***....*.........................................
.....................................**..*.**.**........................................
.....................................***.......*........................................
........................................................................................
........................................................................................
........................................................................................
........................................................................................
........................................................................................
........................................................................................
\end{verbatim}



