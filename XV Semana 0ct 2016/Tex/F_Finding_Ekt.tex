\chapter*{Problema F - Finding Ekt}

\begin{center}
  \begin{tabular}{ | l | l | l | }
    \hline
   Tiepo Límite: 2u & Memoria Límite: 512mb & Código Fuente: \texttt{findingekt.\{java|cpp|c|py\}} \\
    \hline
  \end{tabular}
\end{center}

Una vez recogido el dinero, El Mai quiere dirigirse lo más rápido posible a su tienda de celulares de confianza, pero para eso necesita saber el tiempo que le toma llegar de una estación de autobuses de la ciudad a otro. Podemos asumir que en Chilangolandia la ciudad de El Mai, puedes tomar un autobús que te lleva de cualquier estación a cualquier otra (aunque ir directamente no siempre es lo óptimo), además, El Mai tiene tan buena suerte que siempre habrá autobuses disponibles para él y que además siempre tarda $0$ segundos en transbordar(inclusive en la estación de Cuatro Caminos).

\subsection*{Entrada}

La primera línea de entrada será un número $C$ $(1 \leq  C\leq 10)$, que es el número de casos de prueba, después vendrán $C$ casos. Cada caso comienza con una linea con un número $N$ que indica el numero de estaciones en la ciudad. Después vendrán $N$ renglones, cada uno con N números, donde en la primer renglón vendrán todos los números $a_{0j}$ para $0 \leq j < n$, en el segundo vendrán $a_{1j}$ y así sucesivamente. El número $a_{ij}$ indica el tiempo que toma El Mai en llegar de la estación $i$ a la estación $j$ en la ciudad si  va directamente sin pasar por otra estación de la ciudad. Las estaciones en la ciudad están numeradas del $0$ al $N-1$.  Si $i=j$ entonces $a_{ij}=0$. \\
Después vendrá un numero $Q$ ($1\leq Q \leq 100$) indicando el número de consultas que El Mai tiene, y a continuación vendrán $Q$ lineas, cada una con dos números $a$, $b$ ($0 \leq a, b \leq N-1$) indicando que El Mai quiere saber cual es la mínima cantidad de tiempo que le toma para llegar del punto $a$ al punto $b$.

\subsection*{Salida}
Para cada consulta imprime el tiempo mínimo que le tomaría al Mai llegar de una estación $a$ a otra estación $b$.

\subsection*{Limites de los conjuntos de datos}
\begin{itemize}
    \item Pequeño: $ 1\leq N \leq 10 $ , $ 0\leq a_{ij} \leq 100 $ $\quad \quad \;$ $30$ puntos.
    \item Mediano: $ 1\leq N\leq 20 $ , $ 0\leq a_{ij} \leq 1000 $ $\quad \quad$ $30$ puntos.
    \item Grande: $ \,\;\, 1\leq N\leq 50 $ , $ 0\leq a_{ij} \leq 10^{10} $ $\quad \quad$ $40$ puntos.
\end{itemize}

\begin{multicols}{2}
\subsection*{Entrada Ejemplo}
\begin{verbatim}
2
2
0 4
4 0
1
0 1
3
0 9 1
9 0 2
1 2 0
1
1 0
\end{verbatim}
\columnbreak
\subsection*{Salida Ejemplo}
\begin{verbatim}
4
3
\end{verbatim}
\end{multicols}

