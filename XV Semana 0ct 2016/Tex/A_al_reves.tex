\chapter*{Problema A - Al revés}
\Author{Autor: Moroni Silverio Flores - Grupo de Algoritmia}

\begin{center}
  \begin{tabular}{ | l | l | l | }
    \hline
    Tiepo Límite: 2u & Memoria Límite: 512mb & Código Fuente: \texttt{alreves.\{java|cpp|c|py\}} \\
    \hline
  \end{tabular}
\end{center}

Paquita Cabeza estaba aburrida y empezó a hacer muchas preguntas sobre el destino de los hombres. Como por ejemplo:¿Cuántos números capicúas hay en el intervalo $[a,b]$?

Un número entero positivo es capicúa si se puede leer igual de derecha a izquierda  y de izquierda a derecha, por ejemplo 131. Ayuda a Paquita a resolver esta duda que acongoja su alma.

\subsection*{Entrada}

La primera línea de entrada será un número entero $N$ $(1 \leq  N\leq 1000)$, que es el número de casos de prueba, después vendrán $N$ lineas, cada una de las cuales tendrá un par de enteros $a, b$.

\subsection*{Salida}
Para cada par $a, b$ imprime en una linea distinta cuántos números capicúa hay en el intervalo $[a,b]$.


\subsection*{Limites de los conjuntos de datos}
\begin{itemize}
    \item Pequeño: $ 1\leq a \leq b \leq 200 $ $\quad \quad$ $30$ puntos.
    \item Mediano: $ 1\leq a \leq b \leq 500 $ $\quad \quad$ $25$ puntos.
    \item Grande: $\,\;\, 1\leq a \leq b \leq 10^6 $ $\quad \quad$ $45$ puntos.
\end{itemize}

\begin{multicols}{2}
\subsection*{Entrada Ejemplo}

\begin{Verbatim}
3
130 132
13 15
20 40
\end{Verbatim}
\subsection*{Salida Ejemplo}
\begin{Verbatim}
1
0
2
\end{Verbatim}
\end{multicols}



