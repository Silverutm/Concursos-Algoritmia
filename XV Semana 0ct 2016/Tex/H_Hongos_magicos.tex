\chapter*{Problema H - Hongos Mágicos}

\begin{center}
  \begin{tabular}{ | l | l | l | }
    \hline
    Tiepo Límite: 2u & Memoria Límite: 512mb & Código Fuente: \texttt{hongosmagicos.\{java|cpp|c|py\}} \\
    \hline
  \end{tabular}
\end{center}


Gracias a tu ayuda El Mai ha decidido que celular comprarse, el problema es que no tiene dinero para hacerlo. \\
El Mai tiene un amigo que es duende, el cual le dijo que en el Camino de la serpiente encontraría las monedas necesarias para comprar su celular. \\
El camino de la serpiente esta divido en segmentos y el duende dio una condición para recoger las monedas, y es que si se encuentra en el segmento $i$, El Mai puede o no tomar la moneda que se encuentra en ese segmento, si la toma avanzará al segmento $i + V_i$ ($V_i$ es el valor de la moneda del $i$-ésimo segmento), y si no la toma avanzará al segmento $i+1$. Únicamente los segmentos numerados entre el 1 y el $N$, inclusive, tienen monedas. Ayuda a nuestro amigo a recoger las monedas.

\subsection*{Entrada}

La primera línea de entrada será un número $C$ $(1 \leq  C\leq 100)$, que es el número de casos de prueba, después vendrán $C$ casos. Cada caso comienza con una linea con un número $N$ que indica la cantidad de segmentos con monedas en el Camino de la serpiente. Después vendrá una linea con $N$ números, $V_i$, $1\leq i \leq N$. El i-esimo numero $V_i$ indica el valor de la moneda en el i-esimo segmento.

\subsection*{Salida}
Para cada caso indica la máxima cantidad de dinero que el Mai puede conseguir.

\subsection*{Limites de los conjuntos de datos}
\begin{itemize}
    \item Pequeño: $ 1\leq N \leq 10 $ $ 1\leq V_i \leq 100 $ $\quad \quad \quad$ $20$ puntos.
    \item Mediano: $ 1\leq N\leq 15 $ $ 1\leq V_i \leq 10^4 $ $\quad \quad \quad$ $25$ puntos.
    \item Grande: $ \,\;\, 1\leq N\leq 10^4 $ $ 1\leq V_i \leq 10^{10} $ $\quad \quad$ $55$ puntos.
\end{itemize}

\begin{multicols}{2}
\subsection*{Entrada Ejemplo}
\begin{verbatim}
2
5
1 1 1 1 1
5
1 3 7 1 1
\end{verbatim}
\columnbreak
\subsection*{Salida Ejemplo}
\begin{verbatim}
5
8
\end{verbatim}
\end{multicols}
