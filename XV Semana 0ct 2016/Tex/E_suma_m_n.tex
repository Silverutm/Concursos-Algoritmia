\chapter*{Problema E - Es lo mismo pero diferente}

\begin{center}
  \begin{tabular}{ | l | l | l | }
    \hline
    Tiepo Límite: 2u & Memoria Límite: 512mb & Código Fuente: \texttt{eslomismo.\{java|cpp|c|py\}} \\
    \hline
  \end{tabular}
\end{center}

Después de que Paquita Cabeza aprendió a dividir entre $24$, decidió enfrentar un reto más grande. Puedes decir si $p^r$ divide a $S$, donde:
$$S=\sum_{k=0}^n \binom{n}{k}m^k$$

\subsection*{Entrada}

La primera línea de entrada será un número $C$ $(1 \leq  C\leq 1000)$, que es la cantidad de casos de prueba, después vendrán $C$ lineas, cada una de las cuales tendrá cuatro enteros $p$ (un primo), $r$, $m$, $n$.

\subsection*{Salida}
Para cada caso imprime en una linea distinta ``Lo lograste'' en caso de que $p^r$ divida a $S$, y ``Sad eyes'' en caso de que no.

\subsection*{Limites de los conjuntos de datos}
\begin{itemize}
    \item Pequeño: $ 2\leq p, r, m, n \leq 10^3 $ $\quad \quad $ $20$ puntos.
    \item Mediano: $ 2\leq p, r, m, n \leq 10^6 $ $\quad \quad$ $25$ puntos.
    \item Grande: $\,\;\, 2\leq p, r, m, n \leq 10^{15} $ $\quad \;$ $55$ puntos.
\end{itemize}

\begin{multicols}{2}
\subsection*{Entrada Ejemplo}
\begin{verbatim}
2
2 2 1 1
3 2 2 2 
\end{verbatim}
\columnbreak
\subsection*{Salida Ejemplo}
\begin{verbatim}
Sad eyes
Lo lograste
\end{verbatim}
\end{multicols}

