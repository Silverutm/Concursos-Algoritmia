\chapter*{Problema B - Buenos Amigos}
%ÁÉÍÓÚÑáéíóúñ  ``  ''
¿Te has dado cuenta que en el salón de clases las personas se juntan por grupitos?

A Joaquina le gusta esa idea, ya que no quiere ser amiga de todos, y se pregunta lo siguiente: ¿Cuál es la máxima cantidad de amistades que puede haber en un salón de $n$ alumnos de tal manera que nadie sea amigo de todos los demás?

Toma en cuenta que si la persona $a$ es amiga de $b$, entonces $a$ es amiga de todos los amigos de $b$ y viceversa.

\subsection*{Entrada}
La primera línea contendrá un número $T$, el número de 
casos de entrada. Posteriormente vendrán $T$ líneas, cada 
una con un entero positivo $n$, el número de personas en el salón.



\subsection*{Salida}
Para cada caso, imprime en una línea distinta la máxima cantidad de amistades que se pueden formar.

\subsection*{Límites de los conjuntos de datos}

\begin{itemize}
    \item Pequeño: $ 1 \leq T \leq 10^2 $, $ 1 \leq n
    \leq 10^2$   $\quad \;\;\;\;\;$ $30$ puntos.
    \item Mediano: $ 1 \leq T \leq 10^3 $, $ 1 \leq n
    \leq 10^4$   $\quad \;\;\;\;\;$ $30$ puntos.
    \item Grande: $ 1 \leq T \leq 10^5 $, $ 1 \leq n
    \leq 10^8$   $\quad \;\; \;\;\;\;\;$ $40$ puntos.
\end{itemize}



\begin{multicols}{2}

\subsection*{Entrada Ejemplo}

\begin{verbatim}
2
3
2
\end{verbatim}

\columnbreak

\subsection*{Salida Ejemplo}

\begin{verbatim}
1
0
\end{verbatim}

\end{multicols}