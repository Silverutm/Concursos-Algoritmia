\chapter*{Problema D - Déjalo a la Suerte}
%ÁÉÍÓÚÑáéíóúñ  ``  ''
Vamos a jugar un juego de azar.

Si quieres participar, debes adquirir una papeleta en
la que podrás seleccionar (o no) los números que 
desees entre $1$ y $n$ y registrarlos. Puedes selccionar
la cantidad de números que quieras, siempre y cuando
sea más de uno y menos de $n-1$.

Como suele suceder con este tipo de juegos, habrá un
anfitrión que sacará de una urna una papeleta ganadora,
y listo, ganará la persona que tenga la papeleta con 
los mismos números seleccionados que los del anfitrión.

Para ganar, únicamente es necesaria la existencia de 
los mismos números del anfitrión, sin importar el 
orden.

Si podemos asegurar que no hay dos papeletas que sean
registradas iguales, ¿cuál es la probabilidad
de que ganes si adquieres $k$ papeletas?

Como la probabilidad puede ser muy muy pequeña, 
necesitamos que multipliques tu respuesta por $10^{15}$.
Tendrás un margen de error de $10^{-5}$.




\subsection*{Entrada}
La primera línea contendrá un entero positivo $T$ (el 
número de casos a leer). Luego vendrán $T$ líneas, cada
una compuesta por dos enteros positivos separados por 
un espacio: $n$ y $k$, en ese orden.




\subsection*{Salida}
Para cada caso imprime en una línea distinta la 
probabilidad que tengas de ganar.                   




\subsection*{Límites de los conjuntos de datos}

\begin{itemize}
    \item Pequeño: $ 1 \leq T \leq 100 $, $4 \leq n 
    \leq 10$, $k < n$ $\quad \;\;\;$ $30$ puntos.
    \item Mediano: $ 1 \leq T \leq 100 $, $4 \leq n
    \leq 20$, $k < n$   $\quad \;\;\;$ $20$ puntos.
    \item Grande: $ \;\, 1 \leq T \leq 100$, $ 1 
    \leq n \leq 60 $, $k < n$ $\quad \;\;\;$ $50$ puntos.
\end{itemize}



\begin{multicols}{2}

\subsection*{Entrada Ejemplo}

\begin{verbatim}
3
4 1
10 3
18 6
\end{verbatim}

\columnbreak

\subsection*{Salida Ejemplo}

\begin{verbatim}
166666666666666.66667
2994011976047.90419
22891501911.44041
\end{verbatim}

\end{multicols}