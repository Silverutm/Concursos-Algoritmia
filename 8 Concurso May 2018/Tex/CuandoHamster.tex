\chapter*{Problema C - Cuando una Hámster Quiere Jugar}

A mi hámster le encanta jugar. Hoy decidió salir al
patio a escavar hoyos para matar el aburrimiento. Para 
su comodidad decidí trazar líneas en el patio de tal 
forma que éste fuera una cuadrícula, y en cada uno de 
los cuadros haya una piedra en algún punto que no le 
permitirá a la roedora hacer un hoyo más profundo.

Mi hámster comienza en el cuadro que más le guste en
ese momento y saca toda su tierra hasta encontrar la
piedra. Luego, se mueve a alguno de los cuadros con los
que comparte al menos un vértice y escarva hasta que se
encuentre con la piedra de ese cuadro, o bien, hasta que
alcance la misma profundidad con la que quedó el cuadro 
anterior, lo que suceda \textit{primero}. La inteligente
roedora continúa con este proceso hasta que se encuentre
con una piedra a nivel del suelo, es decir, a profundidad
0.

Considerando que la hámster puede moverse y regresar a
dónde ella desee y que puede moverse de manera 
horizontal, vertical y diagonal en la dirección que 
quiera, ¿cuál es la máxima cantidad de tierra que puede
sacar?
$$ $$
\textbf{Nota:} Mi hámster tiene un asistente que retira
la tierra en cuanto hace un hoyo, así que no tienes que 
preocuparte por pensar qué sucederá una vez que la 
tierra esté fuera del cuadro.




\subsection*{Entrada}
La primera línea estará compuesta por dos números 
naturales $m$ y $n$, que indicarán el tamaño de la 
cuadrícula de la que dispone mi hámster. Luego, se te
dará una matriz $M$ de enteros no negativos de $m \times 
n$, en la que te indicaremos cuál es la profundidad de la
piedra en ``unidades hamsterunas" (las coordenadas de la
matriz empiezan en 1). Finalmente, encontrarás una línea
compuesta por dos enteros $s_i$ y $s_j$, las coordenadas 
del punto en el que el animalillo decidió comenzar.



\subsection*{Salida}
Imprime, en unidades hamsterunas, la máxima cantidad de
tierra que mi mascota podrá sacar.



\subsection*{Límites de los conjuntos de datos}

\begin{itemize}
    \item Pequeño: $1 \leq m, n \leq 50$, $0 \leq m_{ij} 
    \leq 100$, $1 \leq s_i \leq m$, $1 \leq s_j \leq n$
    $\quad \;\;\;\;\;\;\;$ $35$ puntos.
    \item Mediano: $1 \leq m, n \leq 50$, $0 \leq m_{ij} 
    \leq 10^9$, $1 \leq s_i \leq m$, $1 \leq s_j \leq n$
    $\quad \;\;\;\;\;\;\;$ $20$ puntos.
    \item Grande: $1 \leq m, n \leq 500$, $0 \leq m_{ij} 
    \leq 10^{10}$, $1 \leq s_i \leq m$, $1 \leq s_j \leq 
    n$    $\quad \;\;\;\;\;$ $45$ puntos.
\end{itemize}



\begin{multicols}{2}

\subsection*{Entrada Ejemplo}

\begin{verbatim}
3 3
5 0 5
1 2 1
0 0 5
2 2
\end{verbatim}

\columnbreak

\subsection*{Salida Ejemplo}

\begin{verbatim}
10
\end{verbatim}

\end{multicols}

%%%%%%%%%%%%%%%

\begin{multicols}{2}

\subsection*{Entrada Ejemplo}

\begin{verbatim}
2 5
2 3 4 0 1000
3 2 3 0 1000
2 1
\end{verbatim}

\columnbreak

\subsection*{Salida Ejemplo}

\begin{verbatim}
16
\end{verbatim}

\end{multicols}