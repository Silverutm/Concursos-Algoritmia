\documentclass[letter,10pt]{article}
\usepackage[utf8]{inputenc}
\usepackage[spanish, activeacute]{babel}
\usepackage{geometry}
\geometry{verbose,tmargin=0cm,bmargin=2cm,lmargin=2cm,rmargin=2cm,headheight=0cm,headsep=1cm,footskip=1cm}
\usepackage{graphicx}

\usepackage{fancyhdr}
\pagestyle{empty}

\usepackage{fancyhdr}
\pagestyle{fancy}
\cfoot{6}

%%%%%%%%%%%%%%%%%%%%%%%%%%%%%% Textclass specific LaTeX commands.
\newcommand{\lyxaddress}[1]{
\par {\raggedright #1
\vspace{1.4em}
\noindent\par}
}

%%%%%%%%%%%%%%%%%%%%%%%%%%%%%% User specified LaTeX commands.
\date{}

\begin{document}

\title{Raíces digitales}

\includegraphics[scale=0.6]{logo} \hspace*{9.00cm}
\includegraphics[scale=0.5]{dsc} 
\bigskip
\begin{center}
    \Large Problema D - Raíces digitales
\end{center}

\begin{flushright}
Límite de tiempo: 1 segundo
\par\end{flushright}
\bigskip

\section*{Problema}

Recientemente Manuel Nicolás León descubrió qué es la raíz digital, y decidió compartir su conocimiento contigo.

Digamos que $S(n)$ es la suma de los dígitos de $n$, por ejemplo, $S(4089) = 4+0+9+8=21$, entonces la raíz digital del número n es:

$$\begin{array}{llr}
1. & rd(n)=S(n) & \mbox{si }S(n) < 10 \\
2. & rd(n) = rd(S(n)) & \mbox{si }S(n) \ge 10
\end{array}$$

Por ejemplo, $rd(4098) = rd(21) = rd(3) = 3$.

Manuel le tiene miedo a los números grandes, por eso los números con los que trabajará serán a lo más $10^{100}$. Para todos esos números, Manuel ha probado que $rd(n) = S(S(S(S(n)))) (n \leq 10^{100})$.

Ahora Manuel quiere encontrar números rápidamente dada su raiz digital. El problema es que todavía no ha aprendido a hacer lo que te va a preguntar. Tu tarea es, dados los números $k$ y $d$, encontrar números exactamente de $k$ dígitos (sin ceros al principio), con su raíz digital igual a $d$.

\subsection*{Entrada}

La primera línea tendrá un número $T$ ($T \le 500$) que representa el número de casos de prueba.

Las siguientes $T$ líneas contendrán los casos de prueba, cada caso tendrá 2 números, $k$ y $d$ ($1 \le k \le 100$; $1 \leq d \le 9$).


\subsection*{Salida}

Se imprimirán $2T$ líneas, 2 por cada caso de prueba: la primera línea de cada caso de prueba tendrá el número $n$ más grande tal que $rd(n) = d$, y el número de dígitos de $n$ sea igual a $k$; la segunda línea de cada caso de prueba tendrá el número $n$ más chico tal que $rd(n) = d$, y el número de dígitos de $n$ sea igual a $k$.

Puedes tener la seguridad de que dichos números siempre existen, y son únicos.

El primer dígito de cada número impreso no tiene que ser un $0$.

\newpage

$$$$$$$$

\cfoot{7}

\subsection*{Entrada Ejemplo}
\begin{verbatim}
2
1 3
1 7
\end{verbatim}

\subsection*{Salida Ejemplo}

\begin{verbatim}
3
3
7
7

\end{verbatim}

\noindent \rule[0.5ex]{1\columnwidth}{1pt}


\lyxaddress{Sergio Adrián Lagunas Pinacho - Grupo de Algoritmia Avanzada y Programaci'on Competitiva}
\end{document}
