\documentclass[letter,10pt]{article}
\usepackage[utf8]{inputenc}
\usepackage[spanish, activeacute]{babel}
\usepackage{geometry}
\geometry{verbose,tmargin=0cm,bmargin=2cm,lmargin=2cm,rmargin=2cm,headheight=0cm,headsep=1cm,footskip=1cm}
\usepackage{graphicx}

\usepackage{fancyhdr}
\pagestyle{fancy}
\cfoot{8}

%%%%%%%%%%%%%%%%%%%%%%%%%%%%%% Textclass specific LaTeX commands.
\newcommand{\lyxaddress}[1]{
\par {\raggedright #1
\vspace{1.4em}
\noindent\par}
}

%%%%%%%%%%%%%%%%%%%%%%%%%%%%%% User specified LaTeX commands.
\date{}

\begin{document}

\title{E - Buscando asientoa}

\includegraphics[scale=0.6]{logo} \hspace*{9.00cm}
\includegraphics[scale=0.5]{dsc} 
\bigskip
\begin{center}
    \Large E - Buscando asiento
\end{center}

\begin{flushright}
Límite de tiempo: 1 segundo
\par\end{flushright}
\bigskip

\section*{Problema}

Manuel Nicolás León es un chico muy especial... cuando va a conferencias le gusta que la silla de su izquierda y la silla de su derecha estén desocupadas.

Éste año irá a una conferencia a la cuál irán sólo chicos igual de especiales que él (que les gusta que las sillas adyacentes a la suya estén desocupadas). Para el problema tienes que decir si es posible que todos los chicos, incluyendo Manuel, estén a gusto, ésto es, que todos los chicos tengan libres las sillas adyacentes a la suya.

\subsection*{Entrada}

La primera línea tendrá un número $T$ que representa el número de casos de prueba.

Las siguientes $T$ líneas contendrán los casos de prueba, cada caso tendrá 2 números, $A$ y $B$ (1 $\leq$ $A$ $\leq$ $B$ $\leq$ 500) indicando el número de chicos, y el número de sillas disponibles respectivamente.

Los chicos escogen las sillas uno por uno, escogiendo una silla que tenga sus 2 sillas adyacentes desocupadas (Si es una de las sillas de la orilla, basta con que la única silla adyacente que tiene esté desocupada).

Tu tarea es, dadas éstas condiciones, decir si los chicos se sentirán a gusto o no.


\subsection*{Salida}

Se imprimirán $T$ líneas, una por cada caso de prueba, con alguna de las siguientes palabras: ``Si'' si siempre es posible que todos los chicos estén a gusto, ``No'' si es imposible que todos los chicos estén a gusto, o ``Tal vez'' si depende de cómo se vayan sentando los chicos.



\subsection*{Entrada Ejemplo}
\begin{verbatim}
3
1 3
2 3
3 3
\end{verbatim}

\subsection*{Salida Ejemplo}

\begin{verbatim}
Si
Tal vez
No

\end{verbatim}

\noindent \rule[0.5ex]{1\columnwidth}{1pt}


\lyxaddress{Sergio Adrián Lagunas Pinacho - Grupo de Algoritmia Avanzada y Programaci'on Competitiva}
\end{document}
