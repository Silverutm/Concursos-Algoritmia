\documentclass[letter,10pt]{article}
\usepackage[utf8]{inputenc}
\usepackage[spanish, activeacute]{babel}
\usepackage{geometry}
\geometry{verbose,tmargin=0cm,bmargin=2cm,lmargin=2cm,rmargin=2cm,headheight=0cm,headsep=1cm,footskip=1cm}
\usepackage{graphicx}

\usepackage{fancyhdr}
\pagestyle{fancy}
\cfoot{9}

%%%%%%%%%%%%%%%%%%%%%%%%%%%%%% Textclass specific LaTeX commands.
\newcommand{\lyxaddress}[1]{
\par {\raggedright #1
\vspace{1.4em}
\noindent\par}
}

%%%%%%%%%%%%%%%%%%%%%%%%%%%%%% User specified LaTeX commands.
\date{}

\begin{document}

\title{Fibonacci}

\includegraphics[scale=0.6]{logo} \hspace*{9.00cm}
\includegraphics[scale=0.5]{dsc} 
\bigskip
\begin{center}
    \Large F Fibonacci
\end{center}

\begin{flushright}
Límite de tiempo: 1 segundo
\par\end{flushright}
\bigskip

\section*{Problema}

Manuel Nicolás León es un chico muy curioso, tanto que durante una de sus clases se le ocurrió una sucesión de dígitos parecida a la de fibonacci, la cuál empieza con 2 dígitos, 1 y 1, y para la cual cual(es)quiera dígito(s) que se le vaya a añadir es igual a los dígitos que equivalen a la suma de los valores numéricos de los 2 dígitos anteriores, así el tercer dígito sería 1+1=2, el cuarto sería 1+2=3, el quinto 2+3=5, el sexto 3+5 = 8, y así sucesivamente.

Lo interesante de ésta sucesión empieza del dígito 7 en adelante; dado que 5+8 = 13, los dígitos 7 y 8 serían el 1 y el 3 respectivamente; el dígito que ocuparía la posición 9 sería 1+3=4, y así sucesivamente, dando lugar a algo como ésto:

$$112358134 ...$$

Manuel Nicolás León se pregunta si hay alguna forma de saber cualquier dígito de la sucesión con ayuda de un programa, para lo cuál te ha pedido tu ayuda: tienes que programarlo.

%%El planeta Tierra está en peligro a causa de una lluvia de asteroides, por suerte la humanidad tiene la capacidad de crear grandes bombas destructivas para desaparecer los asteroides que se aproximan a la Tierra. Cada bomba tiene un potencial $P$ y su rango de destrucción es una esfera con radio $P$. Dado que la cantidad de bombas que se poseen no son ilimitadas, entonces te han pedido que hagas un programa que dado un punto $T$ encuentres el número de asteroides que destruiría una bomba que sea arrojada a dicho punto.

\subsection*{Entrada}

La primera línea contendrá un entero $N$ ($1 \le N \le 20$), el número de casos de prueba. Las siguientes $N$ líneas tendrán un número entero positivo $S_i$ ($1 \le S_i \le 10^9$) ($1 \le i \le N$) que representa el dígito requerido.
%%La primera línea contendrá un entero $C$ ($1 \le C \le 100$), el número de casos de prueba. Para cada caso de prueba se tendrán los siguientes datos. La primera línea contendrá cuatro enteros $P$, $T_x$, $T_y$, $T_z$  que representan el potencial de la bomba y las coordenadas del punto $T$, respectivamente. La segunda línea contendrá un entero $N$, el número de asteroides que amenazan la Tierra. Las siguientes $N$ líneas contendrán, cada una, tres enteros $x_i$, $y_i$, $z_i$, que representan la posición del i-ésimo asteroide ($-10^3 \le T_x,T_y,T_z,x_i,y_i,z_i \le 10^3$) ($1 \le P \le 10^3$).

\subsection*{Salida}

Se tendrán que imprimir $N$ líneas, una por cada caso de prueba, todas terminando con salto de línea; cada línea tendrá un dígito $D_i$ ($0 \le D_i \le 9$) ($1 \le i \le N$) que representa el dígito que está en la posición $S_i$ requerido.

%%Para cada caso de prueba deberás imprimir un entero, el número de asteroides que la bomba será capáz de destruir.

\newpage

$$$$$$$$$$$$$$$$


\cfoot{10}

\subsection*{Entrada Ejemplo}
\begin{verbatim}
6
1
2
3
7
8
9

\end{verbatim}

\subsection*{Salida Ejemplo}

\begin{verbatim}
1
1
2
1
3
4

\end{verbatim}

\noindent \rule[0.5ex]{1\columnwidth}{1pt}


\lyxaddress{Sergio Adrián Lagunas Pinacho - Grupo de Algoritmia Avanzada y Programaci'on Competitiva}
\end{document}
