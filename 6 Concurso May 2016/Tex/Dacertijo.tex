\includegraphics[scale=0.6]{logo} \hspace*{9.00cm}
\includegraphics[scale=0.5]{dsc} 
\bigskip
\begin{center}
	\Large Problema D - Descifrando el acertijo del profesor
\end{center}

\begin{flushright}
Límite de tiempo: 1 segundo
\par\end{flushright}

\section*{Problema}

Un día, en la clase de Timmy, el profesor Crocker propuso el siguiente acertijo: Dos amigos matemáticos se encuentran después de un largo periodo de no verse, y entablan la siguiente conversación:

\begin{itshape}
- ¡Cuánto tiempo sin verte! Cuéntame, ¿ya te casaste?

- Sí, desde hace dos años.

- ¡Excelente noticia! ¿Y tienes hijos?

- Tengo tres hermosas hijas.

- ¿Qué edades tienen?

- Pues no te voy a decir sus edades, pero la multiplicación de ellas es 36 y la suma es el número de la casa de enfrente.

El amigo voltea a ver el número y después de 1 minuto contesta: -Lo siento, pero me hace falta información.

- ¡Oh! Tienes toda la razón, me faltó decirte que la más grande toca el piano. 
\end{itshape}

Timmy te ha pedido ayuda con un programa que encuentre todas las formas no equivalentes en que puede factorizarse un número utilizando tres términos.

Se considera que una combinación es equivalente si contiene los mismos números sin importar el orden en que aparecen. Así, $(2,8,4)$ es equivalente a $(8,2,4)$.

\subsection*{Entrada}

La entrada inicia con un número entero $1 \leq N \leq 10^3$ que indica los casos de prueba a seguir. Posteriormente, le siguen $N$ líneas, cada una con un valor entero $1 \leq M \leq 10^5$ que representa el número a factorizar.

\newpage
$$
$$
$$
$$

\subsection*{Salida}
Para cada caso de prueba, imprimir una línea con formato “\texttt{\# M:}”, donde $M$ es el número a factorizar utilizando tres términos, seguido de todas las combinaciones no equivalentes ordenando sus números en forma ascendente y posteriormente en forma lexicográfica.

Para cada una de éstas, utiliza el formato \texttt{(x,y,z)}. 

%\begin{multicols}{2}
\subsection*{Entrada Ejemplo}
\begin{verbatim}
2 
3
36
\end{verbatim}


\subsection*{Salida Ejemplo}

\begin{verbatim}
# 3:
(1,1,3)
# 36:
(1,1,36)
(1,2,18)
(1,3,12)
(1,4,9)
(1,6,6)
(2,2,9)
(2,3,6)
(3,3,4)
\end{verbatim}
%\end{multicols}

\noindent \rule[0.5ex]{1\columnwidth}{1pt}


\lyxaddress{Manuel Alcántara Juárez - Club PU++ Facultad de Ciencias}