\includegraphics[scale=0.6]{logo} \hspace*{9.00cm}
\includegraphics[scale=0.5]{dsc} 

\begin{center}
	\Large Problema B - Buscando Potencias
\end{center}

\begin{flushright}
Límite de tiempo: 1 segundos
\par\end{flushright}
\vspace*{-1cm}

\section*{Problema}
Recientemente, Max se enfrentó a la difícil decisión de saber quién era su persona favorita de Oaxaca, así que decidió ponerles este problema a los dos principales contendientes por el título (Kenny y Sergio) para ver de una vez por todas quién era su favorito. ¿Quién podrá resolverlo primero? Obvio, Sergio\footnote{Sergio es conocido por modificar los problemas cuando nadie lo ve.}. Ahora al problema:

Dado un número entero positivo $n$, expresarlo de la forma $(2^k)\times i$, donde $i$ es un número impar. Si pueden existir muchar formas de expresar $n$ de dicha forma, imprime aquélla en donde $k$ sea el máximo posible.

¿Podrás resolverlo tú?

\textbf{Entrada}\\
La primera línea tendrá un número entero $1\leq T\leq 10^5$, que representará el número de casos de prueba.
Seguirán $T$ líneas, cada una con un caso de prueba que será un número entero $n$ tal que $1 \leq n \leq 10^9$.

\textbf{Salida}\\
Para cada caso de prueba, imprimir $k$ e $i$, además del número de caso de prueba, en el formato del ejemplo.
\begin{multicols}{2}
\subsection*{Entrada Ejemplo}
\begin{verbatim}
7
1
2
8
10
26
8967584
1000000000
\end{verbatim}

\subsection*{Salida Ejemplo}
\begin{verbatim}
Caso #1: (2^0)*1
Caso #2: (2^1)*1
Caso #3: (2^3)*1
Caso #4: (2^1)*5
Caso #5: (2^1)*13
Caso #6: (2^5)*280237
Caso #7: (2^9)*1953125
\end{verbatim}
\end{multicols}
\noindent \rule[0.5ex]{1\columnwidth}{1pt}
\lyxaddress{Manuel Alcántara Juárez - Club PU++ Facultad de Ciencias, modicado por Edgar García - Grupo de Algoritmia}