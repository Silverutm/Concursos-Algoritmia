\includegraphics[scale=0.6]{logo} \hspace*{9.00cm}
\includegraphics[scale=0.5]{dsc} 
\bigskip
\begin{center}
	\Large Problema E - Equipo de Progra
\end{center}

\begin{flushright}
Límite de tiempo: 1 segundo
\par\end{flushright}
\bigskip

\section*{Problema}

Lechuga y dos de sus amigos han decido concursar como equipo en el concurso interno de programación, y para celebrar la ocasión decidieron hacer playeras y numerarlas de acuerdo a una antigua leyenda matemática conocida como Tercia Gemela.

Una tercia gemela son 3 números impares consecutivos que, además, son primos, por ejemplo: 3, 5, 7; sin embargo, esta tercia gemela es muy \enquote{mainstream} y ellos quieren usar otra.
Dado un intervalo [a,b], ¿puedes decir cuántas tercias gemelas hay en ese intervalo?

\subsection*{Entrada}

La primera línea contendrá un entero $1 \leq t \leq 1000$: el número de casos de prueba. La primera línea de cada caso de prueba tendrá dos enteros $ 2 \leq a < b \leq 10^8$,  que denotan los límites del intervalo [a,b].

\subsection*{Salida}

Para cada caso de prueba, deberás imprimir el número de tercias gemelas que hay en ese intervalo.

\begin{multicols}{2}
\subsection*{Entrada Ejemplo}
\begin{verbatim}
2
2 10
10 11
\end{verbatim}


\subsection*{Salida Ejemplo}
\begin{verbatim}
1
0
\end{verbatim}
\end{multicols}


\noindent \rule[0.5ex]{1\columnwidth}{1pt}


\lyxaddress{Moroni Silverio Flores - Grupo de Algoritmia}