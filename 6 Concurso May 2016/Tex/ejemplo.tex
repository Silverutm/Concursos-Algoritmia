\includegraphics[scale=0.6]{logo} \hspace*{9.00cm}
\includegraphics[scale=0.5]{dsc} 
\bigskip
\begin{center}
	\Large Problema Ejemplo
\end{center}

\begin{flushright}
Límite de tiempo: 1 segundo
\par\end{flushright}
\bigskip

\section*{Problema}

Dados 2 enteros $A$ y $B$, debes imprimir la suma de ellos.

\subsection*{Entrada}

Una línea con un entero $T$ que indica el número de casos.
En las siguientes $T$ líneas, aparecerán 2 enteros $A$ y $B$.

\subsection*{Salida}
$T$ líneas con la suma de los enteros $A$ y $B$


\begin{multicols}{2}

\subsection*{Entrada Ejemplo}

\begin{verbatim}

2
2 3
5 8

\end{verbatim}

\subsection*{Salida Ejemplo}

\begin{verbatim}

5
13

\end{verbatim}

\end{multicols}
\newpage

$$$$
$$$$
$$$$

\subsection*{Solución}

Los siguientes códigos resuelven el problema anterior:

\begin{multicols}{2}
\begin{itemize}

\item \bf{C}

    \begin{minted}{cpp}
#include <stdio.h>

int main()
{
    int casos,a,b;
	
    scanf("%d",&casos);
	
    while(casos--)
    {
        scanf("%d %d",&a,&b);
        printf("%d\n",a+b);
    }
	
    return 0;
}    
	\end{minted}
    
\item \bf{C++}

    \begin{minted}{cpp}
#include <iostream>
using namespace std;

int main()
{
    int casos,a,b;

    cin >> casos;

    while(casos--)
    {
        cin >> a >> b;
        cout << a+b << endl;
    }

    return 0;
}
	\end{minted}
 

\item \bf{Java}

    \begin{minted}{java}
import java.util.*;

class Bmain{//Ejemplo del nombre de la clase
//para el problema B

    public static void main(String []args)
    {
        Scanner sc = new Scanner(System.in);
        
        int casos,a,b;
        
        casos = sc.nextInt();
        
        while(casos-- > 0)
        {
            a = sc.nextInt();
            b = sc.nextInt();
            
            System.out.println(a+b);
        }
    }
}
    \end{minted}
\end{itemize}

$$$$
$$$$
$$$$
$$$$
$$$$

\end{multicols}


\noindent \rule[0.5ex]{1\columnwidth}{1pt}


\lyxaddress{Grupo de Algoritmia}