\includegraphics[scale=0.6]{logo} \hspace*{9.00cm}
\includegraphics[scale=0.5]{dsc} 
\bigskip
\begin{center}
	\Large Problema A - Aretes Raros
\end{center}

\begin{flushright}
Límite de tiempo: 7 segundos
\par\end{flushright}
%\bigskip

\section*{Problema}
Liliana es una chica chistosa a la que le gusta usar aretes, pero 
no cualquier tipo de aretes. Ella usa aretes de cuentitas con la siguiente característica: una cuentita cuelga de su oreja; de dicha cuentita cuelgan 
2 cuentitas; de cada una de estas cuentitas 
cuelgan, a su vez, otras 2 cuentitas y así sucesivamente hasta quedarse sin cuentitas,
de manera que de todas las cuentitas cuelgan 2 ó 0 cuentitas, excepto tal vez
una que puede llegar a tener una sola cuentita colgando. Además, la cantidad de niveles del arete debe ser la mínima.

\begin{multicols}{3}

\begin{center}
\begin{forest} 
for tree={circle,draw}
[0 
  [8
    [1
      [2
        [4
	      [3]
        ]
        [5]
      ]
      [6]
    ]
    [9]
  ]
  [7
  ]
]
\end{forest}
\captionof{figure}{Arete normal.}
\end{center}

\columnbreak

\begin{center}
\begin{forest} 
for tree={circle,draw}
[0 
  [8
    [1
      [3]
      [9]
      [6]
    ]
    [2]
  ]
  [7
    [4]
    [5]
  ]
]
\end{forest}
\captionof{figure}{Arete normal.}
\end{center}

\begin{center}
\begin{forest} 
for tree={circle,draw}
[0 
  [8
    [1
      [3]
      [9]
    ]
    [2]
  ]
  [7
    [4
      [6]
    ]
    [5]
  ]
]
\end{forest}
\captionof{figure}{Arete especial.}
\end{center}



\end{multicols}

De entre muchos aretes, debes encontrar cuáles le gustarán a Liliana.

\subsection*{Entrada}

La primera línea de entrada será $1\leq T \leq 100$, que es la cantidad de casos de prueba.
La primera línea de cada caso de prueba será un entero $1\leq N\leq 100000$, que indicará la cantidad
de cuentitas del arete.
Seguirán $N-1$ líneas con $2$ enteros $0 <= u, v < N$, que indicarán que la cuentita $v$ cuelga de 
la cuentita $u$. Además, puedes asumir que la cuentita $0$ no colgará de ninguna otra y 
que todos los demás colgarán de una y sólo una cuentita a la vez, es decir, no habrá 
cuentitas volando ni existirán ciclos. 


$$$$
$$$$
$$$$
$$$$

\subsection*{Salida}
Para cada caso, indica en una línea distinta: \texttt{Especial} o \texttt{Normal} según 
corresponda a dicho arete.

\subsection*{Entrada Ejemplo}
\begin{multicols}{2}
\begin{verbatim}
3
10
0 7
0 8
8 9
8 1
1 2
1 6
2 4
2 5
4 3
10
0 8
0 7
8 1
8 2
1 3
1 9
7 4
7 5
1 6
10
0 8
0 7
8 1
8 2
7 4
7 5
1 3
1 9
4 6
\end{verbatim}
\columnbreak
\subsection*{Salida Ejemplo}
\begin{verbatim}
Normal
Normal
Especial
\end{verbatim}
\end{multicols}


\subsection*{Explicación de la entrada ejemplo}
En el primer caso, el arete no es especial porque su altura podría ser menor; en el segundo caso, la cantidad de cuentitas colgando de $1$ excede la cantidad cuentitas que le gustan a Liliana; el tercer caso sí cumple con los requisitos dichos, por lo tanto es especial.

\noindent \rule[0.5ex]{1\columnwidth}{1pt}


\lyxaddress{Edgar García Rodríguez y David  Felipe Castillo Velázquez - Grupo de Algoritmia}